\documentclass[10pt]{article}
\usepackage{kennyworkman}

\newcommand\restr[2]{{% we make the whole thing an ordinary symbol
  \left.\kern-\nulldelimiterspace % automatically resize the bar with \right
  #1 % the function
  \vphantom{\big|} % pretend it's a little taller at normal size
  \right|_{#2} % this is the delimiter
}}

\title{Notes on Projective Geometry}
\author{Kenny Workman}
\date{\today}

\begin{document}

\maketitle

\section{Basic Definitions}

\begin{definition}
	Let $V$ be a vector space. The projective space $P(V)$ is the set of 1-dimensional vector subspaces of $V$.
\end{definition}

It can be useful to think of projective spaces, at least in the real case, as
bundles of lines that pass through the origin. This reduces our space by a
dimension, motivating the common shorthand for the reals $P^n(R) = P(R^{n+1})$. 

Many sources encourage thinking about the sphere $S^n$ for a projective space
$P^n(\R)$ to ease the visualization of a space of lines. Antipodal points on
the sphere are identified and every such antipodal pair is injective with the
actual elements of the projective space. 

I have found it easier to just think of the vector subspaces the projective
elements represent (projective points and lines are lines and planes through the
origin in $\R^3$) for more natural geometric intuition.

\subsection{Decomposition}

The following decomposition is useful to understand the structure of theses spaces:

\[P(R^n) = R^{n-1} + P(R^{n-1}) \]

Essentially our goal is to take $R^n$ and partition the set of points into
1-dim vector subspaces such that each partition has a nice representation.
Recall:

\begin{definition}
	A representative vector is any of the non-zero vectors from the 1-dimensional subspace corresponding to a point $[v] \in P(V)$.
\end{definition}

Then if $[x] = [\lambda x] = [a]$, $x$ and $a$ are both representatives for the same projective point.

We also want to define the notion of the homogoenous coordinates for each
projective point, which are just the real points that exist in the
corresponding vector subspace.

\begin{definition}
	The homogoenous coordinates for $[v] \in P(V)$ are the set $[(x_0 \cdots x_n)]$ equivalent under scalar multiplication by $\lambda$.
\end{definition}

If we construct a subset of homogoenous coordinates $U_0$ where $x_0 \neq 1$,
notice that each $[(x_0 \cdots x_n)] = [1 \cdots x_n / x_0]$, so $U_0 \cong
R^n-1$. We are left to "partition" the coordinates where $x_0 = 0$, but this is
exactly the set of 1-dimensional subspaces of $V^n-1$, so $P(R^{n-1})$.

\subsection{Applications}


\subsection{Linear Subspaces}

We begin by proving a result from elementary linear algebra.

% IIT Notes - https://web.iitd.ac.in/~ritumoni/lecture%2011%20&%2012.pdf.

\begin{theorem}
	Let $W_1$ and $W_2$ be vector spaces. Then $\dim W_1 + W_2 = \dim W_1 + \dim W_2 - \dim W_1 \cap W_2$ 
\end{theorem}

\begin{theorem}
	In a projective plane $P(V)$, two projective lines, $P(U)$ and $P(U')$, intersect in a unique point.
\end{theorem}
\begin{proof}
	From elementary linear algebra, $\dim V \geq \dim U + U'$. We have shown that $\dim U + U' = \dim U + \dim U' - \dim {U \cap U'}$.
	Then $1 \leq \dim {U \cap U'} \leq 2$. Because $P(U)$ and $P(U')$ are distinct, $\dim {U \cap U'} = 1$. So $P(U \cap U')$ is a projective point.
\end{proof}

It is useful to think about this result using our model of the projective plane as a sphere and using our decomposition.

We can think of projective lines as planes in $\R^3$ that intersect the sphere
in two great circles. These great circles intersect in a pair of antipodal
points, which is a projective line.

Alternatively

% TODO

\subsection{Projective Transformations}

Given a linear transformation $T: V \to W$, we might want to recover a bijective map $\tau: P(V) \to P(W)$.

\begin{definition}
	If $T$ is invertible, $\tau$ is a projective transformation between
	projective spaces.
\end{definition}

This is a map of lines to lines.

It seems natural to define $\tau$ as $[x] \mapsto [T(x)]$ for any $x \in P(V)$.
But notice that because there is no $0$ in $P(W)$ (as the collection of 1-dimensional subspaces), 
$\dim T([x]) = 1$ if $\tau$ is to be well-defined over its codomain. Then $T$
must be invertible.

\begin{note}
	It is not immediately obvious that projective transformations describe
	bijections (or isomorphisms between underlying vector spaces). However, a $T$
	that takes any $U$ in $V$ to $0$ induces an ill-defined $\tau$ (no $0$ in
	$P(W)$). Furthermore, if $T(U) = T(U')$ for distinct $U, U'$, $T$ cannot be
	invertible so there will exist some $U''$ where $T(U'')$ goes to $0$. So the
	requirement that $T$ be invertible is really a requirment that $\tau$ needs
	to be a well-defined map with only 1-dimensional subspaces in its codomain.
\end{note}

\begin{note}
	Projective transformations are also called
	\textit{homographies} and have roots
	in the non abstract origins of projective geometry as a tool to study
	perspective. In broad strokes, a homography just describes a
	transformation of perpsectives of the same underlying object.
\end{note}

In fact, projective transformations descibe a collection of linear
transformations $T$ that are equivalent up to scalar multiplication.

\begin{proposition}
	If $T, T': V \to W$ define the same projective transformation, $T = \lambda T'$.
\end{proposition}

\begin{proof}
	If $V$ is generated by basis $\{ v_1 \cdots v_n \}$, then $[Tv_i] = [T'v_i]$
	by assumption. Certainly for each basis element, $Tv_i = \lambda_i T'v_i$.
	For an arbitrary element, $\sum Tv_i = \sum \lambda_i T'v_i$, our assumption tells us $T(\sum v_i) = \lambda
	T'(\sum v_i)$.
	Then, by linearity: \[ \lambda T'(\sum v_i) = T(\sum v_i) = \sum Tv_i = \sum \lambda_i T'v_i\].
	So $\lambda = \lambda_i$. Because $\lambda x = \lambda \sum v_i$, $T = \lambda T'$.
\end{proof}

In the real projective plane, there is very a natural geometric picture one can
construct to see these transformations are bundles of lines related by a
disjoint "observing point".

\begin{example}
Consider two projective lines $P(U)$ and $P(U')$ in the projective plane
$P(V)$. If there exists a point $K \in P(V)$ disjoint from both lines, this
point induces a natural projective transformation $\tau$. For each point (line) in
$P(U)$, draw a line through $K$, and where it meets $P(U')$ is its image under
$\tau$.

We can see this is indeed a projective transformation by proving the underlying
linear transformation $T: U \to U'$ is invertible. If $W$ is the subspace
corresponding to $K$, any $a \in U$ can be uniquely expressed as $w +
a'$ from $W \bigoplus U'$ ($W$ is disjoint from both $U$ and $U'$). Then $a' =
a - w$ where the $w$ component guarantees that $\ker T = 0$.

\end{example}

\begin{note}
	As outlined \href{https://en.wikipedia.org/wiki/Homography#Geometric_motivation}{here},
visualizing the above example in $\R^2$ leads to a natural image of an observer
(our extra point!) connecting two planes together using their perspective. In
fact, this transformation is also called a perspectivity in computer graphics for this
reason.
\end{note}

In vanilla linear algebra, we can fully characterize a linear transformation from an
$n$ dimensional space by observing what it does to $n$ linearly independent
vectors. 

\begin{definition}
	 Points $X_1 \cdots X_{n+1} \in P(V)$ are in general position if any subset
	 of $n$ points have representative vectors that are linearly independent.
\end{definition}

\begin{theorem}
	 If $X_1 \cdots X_{n+2} \in P(V)$ (in $n$ dimensional $P(V)$) are in general
	 position in $P(V)$ and $Y_1 \cdots Y_{n+2}$ are in general position in
	 $P(W)$, then there is a unique projective transformation such that
	 $\tau(X_i) = Y_i$.
\end{theorem}

\begin{proof}

	We can choose representatives, $v_i \in V$, such that $n+1$ representatives
	form a basis of $V$. We can choose representatives such that
	$v_{n+2} = \sum_{i=0}^{n+1} v_i$. Note that the sum of vectors is unique by
	linear indpendence and must exist because $( v_i )_i$ form a basis for $V$.

	Similarly, we can choose representatives from $W$ such that $w_{n+2} =
	\sum_{i=0}^{n+1} w_i$. Again this sum of elements is unique.

	Then there exists a unique and invertible $T: V \to W$ described by the mapping of basis
	elements (where $T(v_i) = w_i$) that induces a projective transform with the
	desired properties

	To see uniqueness, consider an alternative $T': V \to W$ such that $T'(v_i) =
	\mu_i w_i$, taking our basis elements to a different representative of a point in $P(W)$.

	Then $T'(v_{n+2}) = \mu_{n+2} w_{n+2} = \sum_{i=0}^{n+2} \mu_i w_i =
	\sum_{i=0}^{n+2} T'(v_i)$. Because $w_{n+2}$, is the unique sum of
	representatives expressed earlier, $\frac{\mu_i}{\mu_{n+2}} = 1$. So $\mu_i =
	\mu_{n+2}$ and $T = \mu_{n+2}T'$.

\end{proof}

\begin{note}
Any two distinct points on the projective line are linearly independent
vectors, so any three distinct points on the projective line are in general
position.
\end{note}

\begin{theorem}[Desargues' Theorem]
	Consider points $A, A', B, B', C, C' \in P(V)$, where the lines $AA', BB',
	CC'$ are distinct and concurrent (intersect in a single point). Then the
	three points of intersection $AB \cap A'B', BC \cap B'C', CA \cap C'A'$ are
	collinear.
\end{theorem}

\begin{proof}
	Denote the point of intersection between our three lines $P$. Because any
	three points on the projective line are in general position, we can express
	$p$ as three different linear combinations of vectors:

	\[ p = a + a' = b + b' = c + c' \]

	Where $a$ is a representative from $A$ and so forth in the obvious way. We
	can make new points.

	\[ c'' = a - b = b' - a' \]
	\[ a'' = b - c = c' - b' \]
	\[ b'' = c - a = a' - c' \]

	Notice that all of $a'', c'', b''$ lie in a two-dimensional vector
	subspace and are representatives of projective points at our desired
	intersections: $[c''] \in AB \cap A'B'$ and so on. 

	Furthermore, notice $c'' + a'' + b'' = 0$, and this expression is $0$ for any
	linear combination of these vectors. To see this, consider 
	\[\lambda_c c'' + \lambda_a a'' + \lambda_b b'' = 0\] 
	and expand this expression to 
	\[\lambda_c (a - b) + \lambda_a (b - c) + \lambda_b (c - a) = 0\].
	\[(\lambda_c - \lambda_b)a + (-\lambda_c + \lambda_a)b + (\lambda_a + \lambda_b)c\] 
	We can then see we need to choose new representatives from our projective
	lines that will lead to the desired expression.
	\[ p = (\lambda_c - \lambda_b)a + a' = (-\lambda_c + \lambda_a)b + b' =(\lambda_a + \lambda_b) c + c' \]

	Because $a'', b'', c''$ are linearly independent and each exist in a
	two-dimensional subspace of $V$, their corresponding points in $P(V)$ are in
	general position. Three projective points in general position define a line,
	so these points are collinear as desired.

\end{proof}

\subsection{Duality}

We first review definitions of duality from linear algebra.

\begin{definition}
For a vector space $V$ defined over field $F$, the dual of $V$ is the vector
space $V'$ with elements that are linear transformations $f: V \to F$.
\end{definition}

\begin{definition}
If the basis of $V$ is $\{ v_1 \cdots v_n \}$, $V'$ has a corresponding basis
where for each $v_i$, $f_i(v_i) = 1$ and $f_i(v_j) = 0$ for all $j \neq i$.
\end{definition}

Given a linear transformation $T: V \to W$ over vector spaces, there is a
canonical linear transformation induced over its duals $T': W' \to V'$,
given by $T'(f) = f(T)$. Indeed $T'f = fT: V \to F$ and $T'f = fT = v_i \mapsto
f(T(v_i))$.

\begin{note}

We can use the language of contravariant functors to illustrate the
correspondance between linear transformations in their vector and dual spaces.

% TODO commutative diagram

The correspondence of a category of functions to another category with the
domain and codmain swapped is ubituiquitous in category theory and is used to
introduce covariant and contravariant functors in Leinster. 

% If $A \in Set$, $F: [A, A] \to [A, A]^{op}$ k \to Set^{op}$ where if $f: X \to Y$, $Ff = 

\end{note}

\begin{definition}
	The dual of a dual, denoted $V''$, is a vector space of linear
	transformations $(V \to F) \to F$. This space is isomorphic to $V$.
\end{definition}

To see this isomorphism, take $S: V \to V''$, $Sv = f \mapsto f(v)$.
To see it is linear in $v$:

\[S(\lambda_1 v_1 + \lambda_2 v_2)\] 
\[= f \mapsto f(\lambda_1 v_1 + \lambda_2 v_2)\]
\[= f \mapsto (\lambda_1 f(v_1) + \lambda_2 f(v_2))\]
\[= \lambda_1 (f \mapsto f(v_1)) + \lambda_2 (f \mapsto f(v_2))\]
\[= \lambda_1 S(v_1) + \lambda_2 S(v_2)\]

Because $f$ is itself a linear transformation. The kernel of $S$ is $0$ as
there exist no non zero vector $v$ where $f(v) = 0 \forall f \in V'$. Then $S$
is a linear transformation over spaces with the same dimensions and a non-zero
kernel, so it is an isomoprhism.

\begin{definition}
	The annihilator $U^{oo}$ for some subspace $V \subseteq U$ is defined $\{ f'
	\in V' ~|~ f'(f) = 0, \forall f \in U^o \}$
\end{definition}

Then we claim $S(U) = U^{oo}$ for any subspace $U$. Indeed for any $u \in U$,
$Su(f) = f(u)$ for all such $f \in U^o$, so $S(U) \subseteq U^{oo}$.
Equality follows because $S$ is an isomorphism. (The notes claim this follows
from the theorem $\dim V = \dim U + \dim U^o$, but I fail to see this)

This fact will be useful in the next section.

\subsection{Projective Dual Spaces}

If there exists some dual $V'$ for each $V$, we want to say something about
$P(V')$ in its relation to $P(V)$ and maybe use this structure to prove
theorems in the projective space using the dual projective space. 

First, examine a point $f \in P(V')$ and notice $\ker f$ is some $U \subseteq
V$ with $\dim U = n-1$. This point then defines, up to scalar multiplication of
$f$, a unique subspace (ney hyperplane) of $P(V)$.

This gives us a correspondence between the normal vector and dual space that is
not possible in the non projective spaces.

\begin{definition}
	A hyperplane in $P(V)$ of dimension $n$ is some $P(U)$ of dimension $n-1$.
\end{definition}

\subsection{The Fundamental Group of the Projective Plane}

\subsection{Exercises}

Solutions to selected exercises from Hitchin's notes.

\begin{exercise}

	Denote the set of 1-dimensional subspaces of $\{ \operatorname{span} \{x, y\} | x \in U_1,
y \in U_2 \}$ as $A$. Denote the set of 1-dimensional subspaces of $U_1 +
U_2$ as $B$.

To see $A \subseteq B$, for $a \in A$, a one-dimensional subspace of
$\operatorname{span} \{x, y\}$ can be represented by $x + y$, and this is clearly a
one-dimensional subspace of $U_1 + U_2$.

To see $B \subseteq A$, choose some basis vector from $b \in B$, then $b = x'
+ y'$, then it is a one-dimensional subspace of some element of $A$ as
desired.

Note that $A$ is exactly the set of projective points defined by all lines
through $X \in P(U_1)$ and $Y \in P(U_2)$. And $B$ follows from the definition
of $P(U_1 + U_2)$. Their equivalence proves our result.

\end{exercise}

\subsection{Appendix}

\begin{theorem}
	$\dim W_1 + W_2 = \dim W_1 + \dim W_2$
\end{theorem}

\begin{proof}
	Let $S = \{ u_1 \cdots u_r \}$ be the basis of $W_1 + W_2$. Let $B_1 = \{ u_1 \cdots u_r v_1 \cdots v_s \}$ and $B_2 = \{ u_1 \cdots u_r w_1 \cdots w_t \}$ be $B$ extended to be the basis of $W_1$ and $W_2$ respectively. 
	If we can show $B$ is the basis of $W_1 + W_2$, we have our result, as $\dim B = r + s + t = (r + s) + (r + t) - r = \dim W_1 + \dim W_2 - \dim W_1 \cap W_2.$

	First, we show $B$ is linearly independent. Let 

	\[\sum_i^r{a_iu_i} + \sum_j^s{b_jv_j} + \sum_k^t{c_kw_k} = 0\]. 

	Notice if we move terms so 

	\[\sum_i^r{a_iu_i} + \sum_j^s{b_jv_j} = -\sum_k^t{c_kw_k}\], 

	then the LHS is in $W_1$ and the RHS is in $W_2$, so both sides represent the same element in $W_1 + W_2$. 

	Then $\sum_i^r{d_iu_i} = -\sum_k^t{c_kw_k}$, where the LHS uses $B$ and the RHS uses $B_2$. Again moving terms:

	\[\sum_i^r{d_iu_i} + \sum_k^t{c_kw_k} = 0\]

	Where all $c_i$ must be 0 as $B_2$ is linearly independent. Then

	\[\sum_i^r{a_iu_i} + \sum_j^s{b_jv_j} = 0\]

	But the LHS is described by $B_1$ which is also linearly independent so all $a_i$, $b_j$ must also be 0. Then $B$ is linearly independent.

	Consider any $w_1 + w_2$. 

	\[w_1 = \sum_i^r{a_iu_i} + \sum_j^s{b_jv_j}\]
	\[w_2 = \sum_i^r{d_iu_i} + \sum_k^t{c_kw_k}\]

	Then

	\[w_1 + w_2 = \sum_i^r{(a_i + d_i) u_i} + \sum_j^s{b_jv_j} + \sum_k^t{c_kw_k} \in \operatorname{span} W_1 + W_2 \]

\end{proof}

\begin{note}[Spherical, Cyndrical, Cartesian coordinates]

	Cyndrical coordinates of a point in $\R^3$ are simply polar coordinates with
	a z-axis. $(\theta, r, z)$:

	\[ x_1 = r \cos \theta \]
	\[ x_2 = r \sin \theta \]
	\[ x_3 = z \]

	Spherical coordinates use the same $\theta$ around the x axis but decribe the
	point using a distance from the origin $p$ and an angle from the $z$ axis
	$\psi$. $(\theta, \psi, p)$ is 

	\[\theta = \theta \]
	\[r = \sin \psi \]
	\[z = \cos \psi \]

	Then the cartesian coordinates for a point in spherical coordinates is:

	\[x_1 = \sin \psi \cos \theta \]
	\[x_2 = \sin \psi \sin \theta \]
	\[x_3 = \cos \psi \]

	Where $0 \leq \psi \leq \pi$ and $0 \leq \theta \leq 2\pi$

\end{note}

\begin{note}[Quotient space (linear algebra)]
	Let $V$ be a vector space over the field $F$. If $N$ is a subspace of $V$, $V
	/ N$ is the set of equivalence classes where $x \equiv y$ if $x = y + n$ for
	some $n \in N$.
\end{note}

\begin{note}[Small theorem on isomorphisms]
	\begin{theorem}
	$T: V \to W$ where $\dim V = \dim W$. If $ker T = 0$, $T$ is an isomorphism. 
	\end{theorem}
		\begin{proof}
	$T$ is injective. $T(v) = T(v') \implies T(v) - T(v') = 0 \implies T(v - v') = 0 \implies v - v' = 0 \implies v = v'$.
	$T$ is surjective. $\dim V =  \dim \ker T + \dim \im T$. Then $n = 0 + \dim \im T$ so $\im T = W$.
		\end{proof}
\end{note}
	
\end{document}
