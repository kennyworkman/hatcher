\documentclass[10pt]{article}
\usepackage{kennyworkman}

\newcommand\restr[2]{{% we make the whole thing an ordinary symbol
  \left.\kern-\nulldelimiterspace % automatically resize the bar with \right
  #1 % the function
  \vphantom{\big|} % pretend it's a little taller at normal size
  \right|_{#2} % this is the delimiter
}}

\title{Hatcher: Algebraic Topology}
\author{Kenny Workman}
\date{\today}

\begin{document}

\maketitle

\section{The Fundamental Group}

\subsection{Intuition}

\textbf{The Borromean rings} are three linked circles where any two circles without a third are unlinked. It is a nice object that elucidates the difference between abelian and nonabelian fundamental groups.

Label the three rings $A$, $B$, and $C$. Consider $C$ as an element of $\R^3 \setminus A \cup B$. $C$ can be represented as $aba^{-1}b^{-1}$, where $a$ and $a^{-1}$ are forward and reverse oriented paths around $A$ (and $B$). This element is not trivial, so the free group generated by $a$ and $b$ is not abelian.

Modify the ring so that $A$ and $B$ are linked. The same $C = aba^{-1}b^{-1}$ is now trivial. It is easiest to see this visually. Refer to the picture in Hatcher and slide the first loop $a$ all the way around $A$ until it is hanging over the circle. This loop is clearly contractible.

From this, we conclude there are two ways to tell if two circles are linked:

\begin{itemize}
	\item{Show that one of the circles is an element of the fundamental group of the complement of the other circle. (Simple example of loop wrapping around a circle.)}
	\item{Show the fundamental group of the complement of both of the circles is the abelian. (As in the Borromean ring example.)}
\end{itemize}

\subsection{Basic Constructions}

\begin{definition}[Path]
A path in the space $X$ is a continuous map $f: I \to X$ where $I$ is the unit interval $[0, 1]$.
\end{definition}

\begin{definition}[Homotopy of paths]
A continuous family of maps represented by the continuous map $f: I \times I \to X$ 
\end{definition}

\begin{example}[Linear homotopy]
Consider two paths $f, g$ that share endpoints ($f(0) = g(0) = x_0$ and $f(1) =
g(1) = x_1$). Any such paths are always homotopic by the homotopy 
\[
h_t(s) = (1-t)f(s) + (t)g(s).
\]

% TODO: show continuity with vector argument

\end{example}


\begin{proposition}[homotopy of paths with fixed endpoints is an equivalence relation]
\end{proposition}

\begin{definition}[The product of paths]

	For paths $f, g$ where $f(1) = g(0)$, we can define the product $h = f \cdot g$ as

\[ h(s) = \begin{cases} 
      f(2s) & 0 \leq s \leq 0.5 \\
      g(2s-1) & 0.5 \leq s \leq 1
   \end{cases}
\]

\end{definition}


\begin{proposition}[The fundamental group is in fact a group]
	$\pi_1(X, x_0)$ with respect to the operator $[f]\cdot[g] = [f\cdot g]$ is a group.
\end{proposition}

\begin{definition}[Reparamterization of a path]
	We can precompose any path $f$ with $\phi: I \to I$ such that $f\phi \simeq f$. $f\phi$ is then the reparameterization of $f$.
\end{definition}

\begin{proof}
	To see our operator is well-defined, $[f\cdot g]$ depends only on $[f]$ and $[g]$. In other words, a homotopy exists between paths in $[f\cdot g]$ if and only if it is the composite of homotopies between paths in $[f]$ and $[g]$.

	First, we verify associativity of the operator. Consider $(f \cdot g) \cdot h$. This product composes the composition of $f$ and $g$ with $h$. Visually, $h$ is the first half and $g$ and $f$ occupy the last quarters of the product path. We can construct a piecewise function to change the proportions of these components - "shrink" $h$, keep $g$ the same and "expand" $f$. $[(f \cdot g) \cdot h]\phi \simeq f \cdot (g \cdot h)$ gives us the desired equivalence.

	Second, we check the identity property of the constant map $c$. $f$ can be reparameterized by the piecewise function that speeds it up for the first half of $I$ and holds it constant at $f(1)$ for the second half where $f\phi \simeq f \cdot c$.

	Third, we verify the existence of an inverse for each $[f]$. Define $[\bar{f}]$ where $\bar{f}(s) = f(1-s)$. Then $f\cdot\bar{f}$ is homotopic to the constant path. To see this construct $h_t = f_tg_t$ where $f_t = f$ on $[0, t]$ and $f_t = f(t)$ on $[1-t, 1]$ and $g_t$ is the inverse of this function. As $t$ approaches $1$, $h_t$ has a larger constant region in its image starting from the middle and growing to the endpoints until $h_1(s) = f(0) = g(1)$ is our constant map at timepoint 1.

\end{proof}

\begin{proposition}[The fundamental group is independent of the choice of basepoint up to isomorphism]
	If $f$ is a loop with basepoint $x_1$ and $h$ is a path from $x_0$ to $x_1$. The map $\beta_h: \pi_1(X, x_0) \to \pi_1(X, x_1)$ defined as $\beta_h([f]) = [h \cdot f \cdot \hat{h}]$ is an isomorphism.
\end{proposition}
\begin{proof}
	$\beta_h$ is well-defined. To see this, if $[f] = [f']$ implies that $f$ and $f'$ are homotopic, so certainly $h \cdot f \cdot \bar{h}$ and $h \cdot f' \cdot \bar{h}$ are homotopic and $\beta_h([f]) = \beta([f'])$.
	$\beta_h$ is a homomorphism. $\beta_h([f \cdot g]) = [h \cdot f \cdot g \cdot \bar{h}] = [h \cdot f \cdot \cdot h \cdot \bar{h} \cdot g \cdot \bar{h}] = [h \cdot f \cdot \cdot h] \cdot [\bar{h} \cdot g \cdot \bar{h}] = \beta_h([f]) \cdot = \beta_h([g])$. 
	$\beta_h$ is an isomorphism.

\end{proof}

\begin{definition}
	A space is \textbf{simply connected} if it is path connected and has a trivial fundamental group.
\end{definition}

\begin{definition}{lift}
	Given $f: Y \to X$ and a covering space $p: \tilde{X} \to X$, a lift of $f$ is a map $\tilde{f}: Y \to \tilde{X}$ that
	satisfies $f = p\tilde{f}$.
\end{definition}

We are about to compute the fundamental group of the circle, but will use
general properties of lifts (and covering spaces) to make the proof efficient.

(a) Given a path $f: I \to X$, and a covering space 
$p$, and some $\tilde{x_0} \in p^{-1}(x_0)$, there
exists a unique lifted path $\tilde{f}$ starting at $\tilde{x_0}$.

(b) Given a homotopy of paths in $X$, a covering space $p$, and some
$\tilde{x_0} \in p^{-1}(x_0)$, there exist a unique homotopy of lifted paths in
$\tilde{X}$ starting at $\tilde{x_0}$.

We can see that both of these statements are specific versions of the more
general statement (c), given a $F:Y \times I \to X$ and a $\tilde{F}: Y \times
\{ 0 \} \to \tilde{X}$ lifting $\restr{F}{Y \times \{ 0\}}$, there exists a homotopy
$\tilde{F}: Y \times I \to \tilde{X}$ that lifts $F$ and restricts to the given $\tilde{F}: Y \times \{ 0 \} \to
X$. We'll prove (c) so we can use it later.

\begin{proof}
	We construct a lift of an arbitrary neighborhood $N \times I$.

	Consider a fixed $\{y\} \times I$ and notice that for each
	point $(y, t)$,  $F((y, t))$ lies in some evenly covered open set
	$U_t$.

	Because $F$ is continuous, we can construct an open neighborhood $N_t
	\times (a_t, b_t)$ of $y$ where $F(N_t \times (a_t, b_t)) \subseteq U_t$.

	$\{ N_t \times (a_t, b_t) \}_t$ cover $\{y\} \times I$. Since $I$ is compact,
	we can recover a finite subcollection. Let $N$ be the intersection of all the $N_t$ in
	this subcollection and partition $I$ into $\{ [0, t_1] \cdots [t_i, t_{i+1}]
	\cdots [t_m, 1] \}$ in such a way where each closed interval lies within an
	open set in our subcollection. This partition need not be finite and probably
	is not if all of these closed intervals must lie entirely within the
	previously defined open sets.

	Then $N \times [t_i, t_{i+1}]$ each contain $(y, t)$ for some $t \in [t_i,
	t_\{i+1\}]$, and fixed $y$, while $F(N \times [t_i, t_{i+1}]) \subseteq U_i$,
	an evenly covered open set. Because each $U_i$ is homeomomorphic with some
	sheet $\tilde{U}_i$ by $p$, Starting with $N \times [0, t_1]$, while we are
	only given $\tilde{F}$ for $N \times \{0\}$, we can extend the map to the rest of
	this neighborhood by $Fp^{-1}$ for $p$ defined over the sheet that contains
	$\tilde{F}(N \times \{0\})$. We proceed by induction, defining $\tilde{F}$
	over $N \times [t_i, t_{i+1}]$ using $Fp^{-1}$ over the same sheet
	$\tilde{U_i}$ ($t_i$ is shared by the last neighborhood and all neighborhoods
	must lie entirely in some sheet, so they both lie in the same sheet). The
	only additional adjustment might be restricting $N$ for $N \times [t_i,
	t_{i+1}]$ because it was only restricted sufficiently to lie within some
	$U_i$, but not the one homemomorphic to the sheet that we need (?).

\end{proof}

To see that (c) implies (a), consider a path as a homotopy of points. To see
the same with (b), if we are given a homotopy of paths and the starting point
of their lifted paths, the full lifted paths are uniquely determined by (a). If
we consider $Y$ as $I$, a homotopy of the lifted paths follows.

\begin{theorem}[$\Z \cong \pi_1(S^1)$]

	The fundamental group of $S^1$ is the infinite cylical group generated by the
	loop $\omega(s) = (\cos 2\pi s, \sin 2 \pi s)$
\end{theorem}

\begin{proof}

	First recognize that because $[\omega_n] = [\omega]^n$, where $\omega_n = z
	\mapsto (\cos 2\pi nz, \sin 2\pi nz)$, our proof is
	equivalent to showing any loop $f$ is homotopic to some $[\omega_n]$ because
	it is then a product of our generator.

	Let $p: \R \to S^1$ be a covering space defined as $p = s \mapsto (\cos 2\pi
	s, \sin 2\pi s)$. We will first state two useful properties of lifts with
	this covering space and use these to show our result without proof before
	proving them.

\end{proof}

We can immediately use $\pi_1(S^1)$ to prove important theorems. The big idea is that each full rotation around the circle is a unique element of the group. This suggests, among other things, that a loop once around cannot be continuously deformed to a loop twice around.

Our arguments proceed by contradiction, by assuming that our desired result does not hold and showing that this assumption implies some homotopy between loops on the circle that is not allowed.

\begin{theorem}[Fundamental Theorem of Algebra]
Every non constant polynomial with coefficients in $\C$ has at least one root in $\C$.
\end{theorem}

\begin{proof}
	Assume our polynomial $p(z) = z^n + a_1z^{n-1} ... a_n$ has no roots.
	Define the following $f_r: I \to \C$ for each real number:
	\[
		f_r(s) = \frac{p(re^{2\pi is}) \backslash p(r)}{\abs{p(re^{2\pi is})} \backslash \abs{p(r)}}
	\]
	We claim this is a loop in the unit circle $S^1$. To see this, compute some values for fixed $r$ and note for any value of $r$, $f_r(0) = 1$, $f_r(1) = 1$.
	Note that $f_0$ is a constant map with value $1$ and is the trivial loop. Then by varying $r$ we obtain a homotopy between any $f_r$ and $0$ with (the embedded) basepoint $(1, 0)$.
	\indent We now show that $p$ must be constant. Choose a large $r$ such that $r \geq \sum_n{\abs{a_n}} \geq 1$. Observe then $r^n = rr^{n-1} \geq (\sum_n{\abs{a_n}})r^{n-1}$.
	This expression motivates a new homotopy for each $f_r$, let $f_{(r, t)}$ be our previous definition with $p_t(z, t) = z^n + t(z^{n-1}a_1 + ... a_n)$ substituted for each $p$. See that $f_t$ has no zeros on the circle of complex values with radius $r$ satisfying our expression (this is why we constructed it at all).
	Note that $f_{(r, 0)} = e^{2\pi stn}$ (check). Then $f_{(r, t)}$ is a homotopy between $f_r$ and homotopy class $[w_n] \in \pi_1(S^1)$.  But $f_r$ is homotopic to $0$. Then $w_n$ is also homotopic to $0$ and $n$ must be 0. Our $p(z) = a_n$ must be constant.
\end{proof}

\begin{theorem}[Brouwer fixed point theorem]
	Any continuous map $h: D^2 \to D^2$ must have a fixed point $h(x) = x$.
\end{theorem}

This theorem was initially proved by a gentleman named L.E.J. Brouwer circa 1910 and seems to be foundational to the rest of algebraic topology and other fields like differential topology.

We now introduce a theorem which proves that there must exist two antipodal
locations on the surface of the earth with the same temperature and pressure
(assuming that temperature and pressure are continuous functions of location).

\begin{theorem}[Borsuk-Ulam theorem]
	A continuous map $f: S^2 \to \R^2$ must have at least one pair of antipodal points $x$ and $-x$ where $f(x) = f(-x)$.
\end{theorem}

Lets build intuition for the two dimensional case by proving the theorem in one
dimensions. For a map $f: S^1 \to \R$, construct continuous $h = f(x) - f(-x)$.
Pick a point $x$, and evaluate $h$ at points $x$ and $-x$ halfway across the
circle from each other. Notice $h(x) = -h(x)$ so there must exist some $y$
where $h(y) = 0 \in [h(x), -h(x)]$ by intermediate value theorem. We proceed
with proof in two dimensions:

\begin{proof}

	Define a path $\eta(s) = (\cos 2\pi s, \sin 2\pi s, 0)$ around the equator of
	$S^2$ and a map $g: S^2 \to S^1$ defined as $g(x) = \frac{f(x) -
	f(-x)}{\abs{f(x) - f(-x)}}$. Let $h = g\eta$. Check that $\eta$ is
	nullhomotopic (one can shrink the equator to one of the poles) and therefore
	$h$ is also nullhomotopic. (If $\eta \simeq 1$ by $\psi: I \times I \to S^2$,
	then $g\eta \simeq g 1$ by $g \psi : I \times I \to S^1$)

	See that because $g(-x) = -g(x)$ and $\eta(s + \frac 1 2) = -\eta(s)$,
	$g\eta(s + \frac 1 2) = g(-\eta(s)) = -g\eta(s)$, so $h(s + \frac 1 2) =
	-h(s)$. We now examine the lift $\tilde{h}$, induced by $p = s \mapsto (\cos
	2\pi s, \sin 2\pi s)$, and see that $\tilde{h}(s+\frac 1 2) = \tilde{h}(s) +x$
	where $x$ are all the values that map $\tilde{h}(s) +x$ opposite to
	$\tilde{h}(s+\frac 1 2)$ under the trigonometric functions composing $p$.
	Then $x = \frac q 2$ where $q$ is an odd integer. (Check this. Hatcher skips
	some steps that most readers would find obvious probably).

	Think about what we are doing here - lifting $h$ from its path on $S^1$ to
	$\R$ - stretching it out "flat" so we can see how much it winds along $S^1$.

	If $\tilde{h}(s + 1/2) = \tilde{h}(s) + \frac q 2$, we can solve for $q$, $q
	= 2(\tilde{h}(s + 1/2) - \tilde{h}(s))$, and see it cannot be a continuous
	function of $s$ unless it is constant because it is constrained to integer
	values. Then $\tilde{h}(1) = \tilde{h}(\frac 1 2) + \frac q 2 = \tilde{h}(0)
	+ q$, so $\tilde{h}$ is a path that travels a distance $q$ on $\R$, so must wind
	around $S^1$ $q$ times (is $q$ times a generator in $S^1$). So $h$ is not
	nullhomotopic because $q$ is not 0.

	But we saw that $h$, as the composition $g \cdot \eta$, is nullhomotopic.
	This is a contradiction.

\end{proof}

% TODO, really see the role of Borsuk Ulam in the above proof

Reflecting on the structure of this proof for a moment, it is the existence of
some continuous map $g: S^2 \to S^1$. Intuition tells us that the nullhomotopy
sliding the equator to a pole should do nothing when embedded in $S^1$

This theorem says a few things. It tells us the surface of a sphere can never
be one-to-one with an embedding in $R^2$, as two of the points always collapse,
so it cannot be homeomorphic with a subspace of $R^2$. This is obvious when you
think of the stereographic projection of the sphere onto the plane, but is hard
to prove in general.

\begin{exercise}[8]
	Does the Borsuk-Ulam Theorem hold for the torus? For some continuous $f: S^1
	\times S^1 \to \R^2$, does there always exist $(x, y)$ and $(-x, -y)$ where $f((x, y)) = f((-x, -y))$.
\end{exercise}

\begin{figure}[ht!]
\centering
\includegraphics[width=90mm]{meridian-longitude-torus.jpeg}
\caption{A refresher on the meridian and longitude of the torus.}
\end{figure}

\begin{proof}
	We can't do the exact same thing as the proof for $S^2$ because loops on the surface of the
	torus are not nullhomotopic.

	Consider the stereographic projection $p$ of the torus resting on the $0XY$
	plane to this plane. This map is continuous. Each point, denoted $(x, y) \in
	S^1 \times S^1$ on the torus that share an image on the plane must lie on the same meridian
	circle, so must share the first coordinate. Then there cannot be antipodal points
	that satisfy $p((x, y)) = p((-x, -y))$. ($x = 0$ does not lie on $S^1$).
\end{proof}

\begin{note}[Explicit parameterization of torus]
The above problem is mostly an exercise in visualizing geometry and explicit
equations are useful to understand the shape under study.

For $0 \leq \theta \leq 2\pi, 0 \leq \phi \leq 2\pi$ and $r, R$ constants
(the radius of the meridian tube and distance from origin to center of the
meridian tube respectively)
\[z = r \sin \theta \]
\[y = (R + r \cos \theta) \sin \phi \]
\[x = (R + r \cos \theta) \cos \phi \]

See the below figure.

Then continuous $(x, y, z) \mapsto (x, y)$, always maps $(x, y, z)$ and $(-x,
-y, z)$ to different points.
\end{note}

\begin{figure}[ht!]
\centering
\includegraphics[width=90mm]{torus_parameterization.jpg}
\caption{Illustration of paramterization of torus}
\end{figure}


\begin{theorem}[Fundamental group of a product space]
	$\pi_1(X \times Y) \cong \pi_1(X) \times \pi_1(Y)$ if $X$ and $Y$ are path connected.
\end{theorem}

\begin{proof}
	Recall a map into a product space $f: Z \to X \times Y$ is continuous iff
	constituent maps $g: Z \to X$ and $h: Z \to Y$ are continuous. Then each loop
	$\omega: I \to X \times Y$ has equivalent pairs of loops $\omega_X: I \to X$
	and $\omega_Y: I \to Y$. Similarly, homotopies of loops the product space
	have equivalent homotopies in the constituent spaces, which we can use to
	show that $\phi: \pi_1(X, Y) \to \pi_1(X) \times \pi_1(Y)$ defined as $[f]
	\mapsto ([g], [h])$ is a group homomorphism.
\end{proof}

\begin{exercise}[1.1.10]
	From the isomorphism $\pi_1(X \times Y, (x_0, y_0)) \cong \pi_1(X, x_0) \times
	\pi_1(Y, y_0)$, it follows that loops in $X \times \{y_0\}$ and $\{x_0\}
	\times Y$ represent commuting elements in $\pi_1(X \times Y, (x_0, y_0))$.
	Construct an explicit homotopy demonstrating this.
\end{exercise}

\begin{proof}
	Consider arbitrary loops $(g, 1): I \times I \to X \times \{y_0\}$ and $(1, h): I
	\times I \to \{ x_0 \} \times Y$. Our goal is to construct the homotopy that
	shows $[(g, 1)] \cdot [(1, h)] = [(1, h)] \cdot [(g, 1)]$.

	Denote our isomorphism as $\phi$, then $\phi([(g, 1)] \cdot [(1, h)]) =
	([g], [1]) \cdot ([1], [h])$. This element is equivalent to $([g \cdot 1], [1
	\cdot h] = ([1], [h]) \cdot ([g], [1])$ by the homotopy $(F_X, F_Y): (I
	\times I \to X, I \times I \to Y)$. Because $\phi([(1, h)]\cdot[(g, 1)]) =
	([1], [h]) \cdot ([g], [1]), F_{X\times Y}: I \times (I \times I) \to X \times
	Y = (t, (s_g, s_h)) \mapsto (F_X((t, s_g)), F_Y((t, s_h))$ is the explicit
	homotopy in $\pi(X \times Y, (x_0, y_0))$ (where $t$ is some point in time
	$I$ and $s_g, s_h$ are points in the domain of loops $g$ and $h$) that shows
	$[(g, 1)]\cdot[(1,h)]\simeq [(1, h)]\cdot[(g, 1)]$. This homotopy induces
	commutativity and can be constructed from arbitrary $(g, 1)$ and $(1, h)$.
\end{proof}

\begin{example}[Torus]
	The fundamental group of the torus can be thought of as a pair of integers. Formally, $\pi_1(S^1 \times S^1) \cong \Z \times \Z$. 
\end{example}

\begin{definition}[Induced homomorphism]
	The map $\varphi: X \to Y$ where $\varphi(x_0) = y_0$ induces a homomorphism $\varphi_*: \pi_1(X, x_0) \to \pi_1(Y, y_0)$ defined as $\varphi_*([f]) = [\varphi f]$
\end{definition}

We can briefly check some properties of this homomorphism. Let $\varphi, \psi$ be maps:
\begin{itemize}
	\item{$(\varphi\psi)_* = \varphi_*\psi_*$}
	\item{$\mathds{1}_* = \mathds{1}$}
\end{itemize}
These properties of the induced homomorphism make the fundamental group a functor.

The induced homomorphism allows us to describe relationships between
topological spaces as relationships between fundamental groups. Showing there
exist continuous maps between shapes of interest and a shape that admits a
trivial fundamental group is then a useful tool in computing the fundamental
group of the original shape. Some examples follow.

\begin{theorem}
	$\pi_1(S^n)$ is trivial if $n \geq 2$
\end{theorem}

\begin{proof}
	We use Lemma 1.15 but omit its proof. Given a loop $f$ with basepoint $x_0$ in $X$, and some
	decomposition of $X$ into open $A_{\alpha}$ where $X = \bigcup_{\alpha}
	A_{\alpha}$, $x_0 \in A_{\alpha}$ and $A_{\alpha} \cap A_{\beta}$ is path
	connected. There exists a homotopy from $f$ to a product of loops in each
	$A_{\alpha}$.

	Consider $S^n = A_1 \cap A_2$ where $A_1 = S^n - \{y\}$ and $A_2 = S^n -
	\{z\}$ for $y, z$ antipodal on $S^n$. $A_1, A_2 \approx R^{n-1}$. Then every
	$f$ in $S^n$ is the product $g \cdot h$, where $g$ and $h$ are loops in $A_1$
	and $A_2$ respectively. Let $\psi: A_1 \to \R^{n-1}, \phi: A_2 \to \R^{n-1}$
	be homeomorphisms, and because $\pi_1(\R^{n-1})$ is trivial, $[g], [h]$ are
	nullhomotopic in $\pi_1(A_1), \pi_1(A_2)$ under $\psi_*, \phi_*$. So $[f] =
	[g] \cdot [h] = 1$ for any $f$.
\end{proof}

\begin{exercise}[1.1.12]
	Show that every homomorphism $\phi: \pi_1(S^1) \to \pi_1(S^1)$ can be realized as the
	induced homomorphism of the map $\varphi: S^1 \to S^1$.
\end{exercise}

\begin{proof}
	We can consider the set of homomorphisms $\{\phi: \pi_1(S^1) \to \pi_1(S^1)\}$
	as the equivalent set of homomorphisms $\{\phi: \Z \to \Z\}$.

	First we see that the set of homomorphisms $\{\phi: \Z \to \Z\}$ are described by
	the set of maps $\{ \phi' = x \mapsto mx ~|~ m \in \Z \}$. This set includes the identity and 0
	map, but also all homomorphisms that scale group elements by an integer factor.

	We then claim the set of maps $\{f_m: S^1 \to S^1 ~|~ m \in \Z\}$ where $f_m = (x, y) \mapsto (mx, my)$
	can be realized as $\{ \phi' \}$ as induced homomorphisms.

	To see this, consider a loop in $S^1$, $\omega_n$, where $n$ denotes the
	number of turns around the circle (formally $s \mapsto (\cos 2\pi ns, \sin
	2\pi ns)$) and examine its transormation under $f_m$. $f_m$ stretches (or
	shrinks) $\omega_n$ to a loop $\omega_{mn}$. This is clear by looking
	at the lifts of $\omega_n$ and $f_m\omega_n$, which are maps over $\R \to S^1$
	defined as $\omega_{n*} = x \mapsto (\cos
	2\pi x, \sin 2\pi x)$ and $f_m\omega_{n*} = x \mapsto (\cos m 2\pi x, \sin m
	2\pi x)$, to see that $f_m$ indeed stretches (or shrinks) a "$n$ turn" loop to
	an "$n*m$" turn loop. Then $f_m* = [\omega_n] \mapsto [\omega_{n*m}]$
	corresponds to a $\phi': x \mapsto mx$ as desired and $\{f_m\}$ and
	$\{\phi'\}$ are in one-to-one correspondence.
\end{proof}

\begin{exercise}[1.1.13]
	Consider a space $X$ and a path connected subspace $A$ with basepoint $x_0
	\in A \subseteq X$. Show that every map $\pi_1(A) \xhookrightarrow{} \pi_1(X)$ is surjective iff each path in $X$ that has its endpoints in $A$ is
	homotopic to a path in $A$.
\end{exercise}

\begin{proof}
	We begin with the reverse direction. A loop $g$ in $X$ with basepoint $x_0$ is
	certainly a path in $X$ with its endpoints in $A$, so it is homotopic to a
	path in $A$ with the same endpoints, a loop, $f$ in $A$. Then for every $\theta:
	\pi_1(A) \xhookrightarrow{} \pi_1(X)$, $[g] \in \pi_1(X, x_0)$ must be equal
	to some $\theta([f])$ as $f \simeq g$.

	To see the forward direction, define $h$ as the path in $X$ with endpoints in
	$A$. We can turn $h$ into a loop with the product $\eta h \bar{\mu}$ where
	$\eta$, $\mu$ are the paths from $x_0$ to the first and second endpoint
	respectively. Recall $A$ is path connected so we can always construct such
	paths. Because all $\pi_1(A) \xhookrightarrow{} \pi_1(X)$ are surjective,
	$\eta h \bar{\mu} \simeq h'$, where $h'$ is a loop entirely contained in $A$.
	If we restrict the homotopy $F_t: I \to X$ to the closed subset of $I$ that
	is $h$ at time 0, the restricted homotopy shrinks $h$ to a path in $A$, as
	desired.

\end{proof}

\begin{proposition}[]
	For a given retraction $r: X \to A$, the induced homomorphism for the associated inclusion $i_*$ is injective. If $r$ is a deformation retraction, $i_*$ is also an isomorphism.
\end{proposition}

\begin{proof}
	Because $ri = \mathds{1}_A$, $r_*i_* = \mathds{1}_{A*}$ so $i_*$ is
	injective. To see $i_*: \pi(A) \to \pi(X)$ is also surjective, see that any
	loop $[x] \in \pi_1(X)$ homotopes to a loop in $A$ by $r_t$ if $r_t$ is a deformation
	retract, so $i_*^{-1}([r_tx]) \in \pi_1(A)$.
\end{proof}

So spaces that are deformation retracts have isomorphic fundamental groups.

Note that deformation retracts are rather strict examples of homotopies as not
only do they fix a basepoint between $X$ and $A$ for all timepoints, but they
restrict to the identity map over their retracting space for all $I$
($\restr{r_t}{A\times I} = \mathds{1}$). 

Lets work towards an induced isomorphism under the more general condition of
homotopy equivalence.

Consider a homotopy $\varphi_t: X \times I \to Y$ that is not a deformation retract but fixes the basepoint in $X$ ($\varphi_t(x_0) = y_0$ for all $t$). Then $\varphi_{0*} = \varphi_{1*}$ as $[\varphi_0f] = [\varphi_1f]$. Any pair of induced homomorphisms under a basepoint preserving homotopy are equivalent.

Now consider homotopy equivalences that are also basepoint preserving. Let $\varphi: X \to Y$ and $\psi: Y \to X$ be such equivalences. Then $\psi\varphi \simeq \mathds{1}$ but we just saw then $(\psi\varphi)_* = \mathds{1}_*$ so $\varphi_*$ is injective. $\varphi_*$ is also surjective as $(\varphi\psi)_*$ = $\mathds{1}_*$

	We can actually see this is true even if our homotopy equivalence does not fix our basepoint.

\begin{lemma}[]
	Let $\varphi_t: X \to Y$ be a homotopy and $\restr{h}{x_0 \times I}$ be a path between $\varphi_0(x_0)$ and $\varphi_1(x_0)$. Then $\varphi_{0*} = \beta_h \varphi_{1*}$
\end{lemma}

\begin{note}
	To see this, for a given loop $f$ in $X$, we can build a new homotopy $\bar{h_t} \cdot \varphi_tf \cdot h_t$. Then $\beta_h(\varphi_0([f]))$ is homotopic to $\varphi_1([f])$.
\end{note}

\begin{theorem}[The choice of basepoint between homotopy equivalent spaces is not important]
Consider the homotopy equivalence $\varphi: X \to Y$, then $\pi_1(X, x_0) \cong \pi_1(Y, \varphi(x_0))$.
\end{theorem}

\begin{proof}
	Consider the homotopy inverse $\psi: Y \to X$. Then $\psi\varphi \simeq \mathds{1}$. We just saw that $(\psi\varphi)_* = \beta_h(\mathds{1})_*$ for some path $h$ from $\psi\varphi(x_0)$ to $x_0$. Then $\psi_*\varphi_* = \beta_h$ is an isomorphism and $\varphi_*$ is injective.
	Using the same argument, $\varphi\psi \simeq \mathds{1}$ and $\varphi$ is also surjective.
\end{proof}

\begin{exercise}[]

\end{exercise}
\begin{proof}
\end{proof}

\begin{exercise}[1.1.3]
	$\pi_1(X)$ is abelian iff the basepoint-preserving homomorphism $\beta_h$ depends only on the choice of endpoints of $h$
\end{exercise}

\begin{proof}

	Consider homotopy classes $[f], [h]$ in $X$. By definition, $\beta_h([f]) = [hfh^{-1}]$. Consider a new constant path $c = s \mapsto h(0)$ (the basepoint of $X$). Then if $\beta_c = \beta_h$, $[f] = \beta_c([f]) = \beta_h([f]) = [hfh^{-1}] = [h][f][h^{-1}]$ for any choice of $f, h$. $\pi_1(X)$ is abelian.

	To see the reverse, consider two arbitrary paths $g, h$ (they need not be loops). Because $\pi_1(X)$ is abelian, $\beta_g([f]) = [g^{-1}fg] = [h^{-1}fh] = \beta_h([f])$ for any $h$.

\end{proof}

\begin{exercise}[1.1.5]
	Let $X$ be some space. The following conditions are equivalent:
	\begin{itemize}
		\item{Any map $S^1 \to X$ is homotopic to a constant map}
		\item{Any map $S^1 \to X$ extends to some map $D^2 \to X$}
		\item{$\pi_1(X, x_0) = 0$ for any $x_0 \in X$}
	\end{itemize}
\end{exercise}

\begin{proof}
	To see $(c)$ implies $(a)$, consider any loop $f$. There exists a single homotopy class. Then $f$ homotopic with the constant loop, $\mathds{1}(I) = x_0$.
	\par Now we show $(a)$ implies $(b)$. Let $f: S^1 \to X$ be our map and $h: S^1 \times I \to X$ be the homotopy relating $f$ to some constant map. Let $g: D^2 \to S^1 \times I$ be a homeomorphism. Then $hg$ is a continuous map where $\restr{hg}{S^1} = f$.
	\par Now we show $(b)$ implies $(c)$. Let $f: S^1 \to X$ be an arbitrary loop. Let $h: S^1 \times I \to D^2$ be a homotopy relating the identify on the circle to a constant map. Let $g$ be the extension of $f$. Then $gh$ homotopes $f$ to a constant map whose image is a point in $X$.
\end{proof}

\begin{exercise}[1.1.6]
	The canonical map $\phi: \pi(X, x_0) \to [S^1, X]$ is onto if $X$ is path connected and places the conjugacy classes of $X$ in one-to-one correspondence with $[S^1, X]$.
\end{exercise}

\begin{proof}
	To see $\phi$ is surjective if $X$ is path connected, consider $[f]$ in $[S^1, X]$. Let $h$ be a path from $x_0$ to any point on the path $f$ (note that $h$ is constant if $x_0$ already lies on $f$). Then $\phi^{-1}([f]) = [\bar{h}fh]$.
	\par A proof of one-to-one correspondence between $[S^1, X]$ and conjugacy classes of $\pi_1$ is equivalent to showing $\phi([f]) = \phi([g])$ iff $[f]$ conjugates $[g]$ in $\pi_1$. 
	\par To see the forward direction, let $\psi_t$ be the basepoint free homotopy relating $\phi([f])$ and $\phi([g])$. Then $\psi_{0*} = \beta_h\psi_{1*}$, where $h$ is the path induced by the image $\psi_t(0)$. Notice that $h$ starts and ends at $x_0$ because $[f]$ and $[g]$ had basepoints of $x_0$ under the preimage of $\psi$, so $h$ is also a loop. Then $\psi_{0*} = \beta_h\psi_{1*}$ is equivalent to $[h \cdot g \cdot \bar{h}] = [f]$ which implies $[h] \cdot [g] \cdot [h]^{-1} = [f]$.
	\par To see the reverse direction, let $[h]$ be the element of $\pi_1$ that conjugates $[f]$ and $[g]$.

\end{proof}

\subsection{Van Kampen's Theorem}

To understand free products, lets review the direct products.

\begin{definition}[Direct product of groups]
	Given an indexed set of subgroups $\{G_{\alpha}\}$, our direct product group $\prod_{\alpha} G_{\alpha}$ is the set of functions $\{ f = \alpha \mapsto g_{\alpha} \in G_{\alpha} \}$.
\end{definition}

Some facts that are useful to verify:

\begin{enumerate}
	\item{$\bigcap G_{\alpha} = 1$. (Every element is unique in the product, even if two copies of the same group are used in the product.)}
	\item{For each factor group, there exists a surjective "projection homomorphism" $\pi_i: G \to G_{\alpha}$ where $G_{\alpha} = \{(1 \dots g_i \dots 1) \}$. }
	\item{ $G \backslash G_{\alpha} \cong \{ (g_1, ... g_{\alpha-1}, g_{\alpha+1}, ...  g_n) \}$ (To see this, construct a homomorphism that erases $\alpha$ component and invoke the First Isomorphism Theorem).}
	\item{Elements of each factor group commute with each other.}
\end{enumerate}

% Refer to 5.1 in Dummit and Foote

\begin{proposition}[Proof that direct products are commutative up to isomorphism.] 
	If $G = H \times K$, then $hk = kh$.
\end{proposition}

\begin{proof}
	Let $\psi: G \to H \times K$ be an isomorphism carrying elements of $G$ to their tuple representation. $\psi(kh) = \psi(k)\psi(h) = (1, k)(h, 1) = (h, k) = \psi(hk).$
\end{proof}

It is helpful to think of each element of the product as embedded in some latent tuple. 

Now we might want to compose a group that is not commutative amongst factors. This will help us distinguish, for example, $\Z \star \Z$ (disjoint circles) from $\Z \times \Z$ (linked circles or torus). We introduce the free group:

To use the language of category theory, the free product is the coproduct in the category of groups.

% TODO
Formally:

% TODO
We show the binary operator of free groups is associative.

\begin{proof}[Associativity of free groups]

\end{proof}

The proof of Van Kampen's also relies on basic facts about homomorphisms and quotient groups. We derive some of these facts here to refresh the dome.

\begin{proposition}[]
	Consider $\phi: G \to H$ where $ker \phi = K$. If $X = \phi^{-1}(a)$ is an element of $G / K$, then for any $u \in X$, $X = \{ uk ~|~ k \in K \}$.
\end{proposition}

\begin{proof}[]
	To see $uK \subseteq X$, observe $\phi(uk) = \phi(u)\phi(k) = \phi(u) \in X$. To see $X \subset uK$, consider $x \in X$ and notice that $\phi(u^{-1}x) = 1$. Then $u^{-1}x = k$ and $x = uk \subset uK$.
\end{proof}

\begin{proposition}[]
	Consider $\phi: G \to H$ where $ker \phi = K$. The group operation in $G / K$ is well-defined.
\end{proposition}

\begin{proof}[]
	Consider arbitrary $X, Y \in G / K$ where $X = aK$ and $Y = bK$. Let $Z = XY$. To see multiplication is independent of choice of representative, pick arbitrary representatives $u \in X$ and $v \in Y$. Then $uv \in \phi^{-1}(ab)$.
\end{proof}

In fact, multiplication of quotient group elements is well defined under a more general condition - that our coset is a normal subgroup.

\begin{proposition}[]
	Let $uH, vH \in G / H$. Then $(uH)(vH) = uvH$ is well defined iff $H \trianglelefteq G$.
\end{proposition}

\begin{proof}[]
	For the forward argument, choose arbitrary $x, x^{-1} \in G$. Then $(xH)(x^{-1}H) = H$, so $(xh)(x^{-1}1) \in H$. Then $H$ is normal as desired.
	For the reverse argument, consider alternative representatives $u' \in uH$ and $v' \in vH$. We must show $u'v' \in uvH$. $u' = um$ and $v' = vn$ for some $m, n \in H$. Then $(um)(vn) = uvv^{-1}mvn$. But $v^{-1}mv \in H$ by assumption (denote this $n'$). Then $uvv^{-1}mvn = uvn'n \in uvH$.
\end{proof}

\begin{theorem}[Van Kampen's Theorem]

	Let $i_{\alpha\beta}: \pi_1(A_{\alpha} \cap A_{\beta}) \to \pi_1(A_{\alpha})$ be the homomorphism induced by the inclusion $A_{\alpha} \cap A_{\beta} \xhookrightarrow{} A_{\alpha}$ and $j_{\alpha}: \pi_1(A_{\alpha}) \to \pi_1(X)$ be the homomorphism induced by $A_{\alpha} \xhookrightarrow{} X$.

	Observe $\phi(i_{\alpha\beta}(\omega)i_{\beta\alpha}(\omega)^{-1}) = j_{\alpha}i_{\alpha\beta}(\omega)j_{\beta}i_{\beta\alpha}(\omega)^{-1} = 1$ because $j_{\alpha}i_{\alpha\beta} = j_{\beta}i_{\beta\alpha}$. (Recall $\phi$ is the inclusion homomorphism applied to each letter in the word). Clearly all such elements of the free group are in the kernel of $\phi$. We will show the normal group generated by such elements are exactly the kernel of $\phi$.

\end{theorem}

\begin{proof}

	To see $N$ is the kernel of $\phi$, we show that $\star_{\alpha} \pi_1(A_{\alpha}) \backslash N \cong \pi_1(X)$.
	Observe $[f_i]_{\alpha}N = [f_i]_{\beta}N$ by the definition of $N$ ($[f_i]_{\alpha}[f_i]_{\beta}^{-1} \in N$). Furthermore $[f_1][f_i]_{\alpha}[f_2]N = [f_1][f_i]_{\beta}[f_2]N$ as $N$ is normal so the group operation is well defined on cosets.

	% explicitly define isomorphism
	% - well defined
	% - homomorphic
	% - injective
	% 
	% show injectivity explicitly 
\end{proof}

\begin{note}

Van Kampen's allows us to treat the fundamental group of wedge sums as the free group of summands.

It gives us the isomorphism $\star_{\alpha} \pi_1(X_{\alpha}) \cong \pi_1(A_{\alpha})$

$\pi_1(X_{\alpha}) \cong \pi_1(A_{\alpha})$

where $A_{\alpha} = U_{\alpha} \bigvee_{\beta \neq \alpha} X_{\beta}$.

% Why is this not trivial? Why construct these neighborhoods?
% https://math.stackexchange.com/questions/4415409/fundamental-group-of-a-wedge-sum
% https://en.wikipedia.org/wiki/Hawaiian_earring
\end{note}

\begin{note}Group descriptions of basic objects
	\begin{itemize}
		\item{$\pi_1(S^1) \cong \Z$}
		\item{$\pi_1(S^2) \cong 0$}
		\item{The fundamental group of a torus is $\pi_1(S^1 \times S^1) \cong \Z \star \Z$}
	\end{itemize}
\end{note}

Van Kampen's can be used to derive a powerful result about 2-dimensional complexes. If we consider the decomposition of such a complex as a 1-skeleton (wedge sum of circles) with a collection of attached 2-cells, then the fundamental group of this complex is the free group on some number of generators corresponding to the number of circles in the 1-skeleton modulo a normal group generated by loops where the 2-cells are attached. Recall 2-dimensional complexes includes quite a broad group of objects we are familiar with, such as the orientable surfaces of genus 2 ($M_2$ being the torus) and the non-orientable surfaces like the mobius strip and klein bottle.

\begin{theorem}[van Kampen's applied to 2-dimensional cell complex]
	Let $X$ be some path connected space and $Y$ be the result of attaching a
	collection of 2-cells $\{e_{\alpha}\}$. Then the isomomorphism induced by the
	inclusion $\pi(X) \to \pi(Y)$ has a kernel that is exactly $N = <
	\bar{\gamma_{\alpha}}\varphi_{\alpha}\gamma_{\alpha} >$. Then, (by van
	Kampen's), $\pi(X) / N \cong \pi(Y)$
\end{theorem}

We can use this theorem to compute the fundamental group of the torus from the abelianization of the free group in two generators. Let $X = S^1 \vee S^1$ and $Y = S^1 \times S^1$ and observe that $X$ is the one-skeleton of the torus and $Y$ is obtained by attaching a 2-cell to this skeleton along the loop $a^{-1}b^{-1}ab$ (trace the edges of a rectangle).

We know $\Z \star \Z / [\Z, \Z]  \cong \Z \times \Z$. Because $\pi(X) \cong \Z \star \Z$ and $\pi(Y) = \Z \times \Z$, it remains to see that $N \cong [\Z, \Z]$ to have our result. 

% TODO show the connection between commmutator subgroup and loops in the complex

\begin{theorem}
	If $g \neq h$, then $\pi_1(M_g)$ and $pi_1(M_h)$ are not isomoprhic. Nor are the spaces homotopy equivalent.
\end{theorem}

\begin{proof}
	If $M_g \simeq M_h$, then their fundamental groups are isomorphic. These groups are the abelianizations of free groups in $2g$ and $2h$ generators respectively (this is immediate from the previous result). Because these groups are isomorphic, $g$ and $h$ must be the same.
\end{proof}

% Problems

We will used the fact that a punctured n-sphere is homeomorphic to the 

\begin{proposition}[Proof of generalized stereographic projection]
There exists a homeomorphism between $S^n-1$ and
\end{proposition}

\begin{exercise}[1.2.4]
	Compute $\pi_1(\R^3 - X)$ where $X$ is the union of $n$ lines through the origin.
\end{exercise}

\begin{proof}
	Observe the complement deformation retracts to $S^2$ with $2n$ points removed (the poles of our $n$ lines). This object is homoemorphic to $\R^2$ with $2n-1$ points removed. (maybe construct the explicit stereographic projection). We can use van kampens to show the fundamental group of this space is the free group of $2n-1$ generators.
\end{proof}

\begin{exercise}[1.2.6]
	Supppose a space $Y$ is obtained from a path-connected space $X$ by attaching
	$n$-cells for a fixed $n \geq 3$. Then show the inclusion $X
	\xhookrightarrow{} Y$ induces an isomorphism on $\pi_1$. Use this result to
	then show	that the complement of a discrete subspace of $R^n$ is simply-connected if $n
	\geq 3$.
\end{exercise}

\begin{proof}

	We proceed using the structure of the proof of $1.26$. Let $A$ be $Y$ with a
	hole in each of the attached $n$-cells. Let $B$ be the union of the attached
	$n$-cells. See that $A$ and $B$ are both open and $Y = A \cup B$, so we
	invoke van Kampen's theorem: $\pi_1(Y) = \pi_1(A) \star \pi_1(B) / N$ where
	$N$ is defined in the usual way.

	As before, $A$ deformation retracts onto $X$, so $\pi_1(A) \cong \pi_1(X)$,
	and $B$ is contractible, so $\pi_1(B) = 0$. Then it remains to show that $N$
	generated by $\{ i_{AB}(\omega) i_{BA}(\omega)^{-1} | \omega \in \pi_1(A \cap
	B) \}$ is trivial (where the map $i_{AB}$ is defined as $i_{AB}: \pi_1(A \cap
	B) \to \pi_1(A)$). See that any loop $i_{AB}(\omega)$ is contractible,
	because despite introducing a hole, the loop can homotope around it within the
	interior of the $n$-cell (if $n \geq 3$). Any $i_{BA}(\omega)$ is trivially
	contractible, being included into an open subset of an $n$-cell with no
	holes. Then $\pi_1(Y) \cong \pi_1(X)$.

	% https://math.stackexchange.com/questions/1188062/why-is-the-complement-of-a-discrete-subspace-of-mathbbrn-n-ge-3-simpl

	To prove our last result, let $X$ be our space of interest (the complement of
	some discrete subspace of $R^n$), we can obtain a covering $\{ A_{\alpha} \}_{\alpha}$
	where each $A_{\alpha} = X - x_{\alpha}$ is the complement of each point in the discrete
	subspace. See each $A_{\alpha}$ is open. Then $\pi_1(Z) = \star \pi_1(A_{\alpha}) / N$.

\end{proof}

\begin{exercise}[1.2.7]
	Construct a cell complex for the two-sphere with north and south poles
	identified and use this to compute the fundamental group of this space.
\end{exercise}

\begin{proof}
	We first construct the cell complex. Begin with a 0-cell $x$ and attach a
	1-cell $e$ to $x$ to form a circle. Then take a 2-disk $\sigma$ and attach
	its boundary circle $\delta \sigma$ to $e$ in two pieces, $\delta_1 \sigma$
	traces one endpoint to the other and attaches to $e$, while $\delta_2 \sigma$
	completes the circle in the same orientation but attaches to $\bar{e}$.

	Then, by van Kampen's, our fundamental group is isomorphic to $< a ~|~ a a^{-1}
	>$, but this is just the free group on one generator, so $\pi_1(X) \cong \Z$.

\end{proof}

Seeing that the cell complex is the same as the sphere with north and south
poles identified was not immediately obvious.

\href{https://math.stackexchange.com/questions/1194611/cw-complex-structure-on-standard-sphere-identifying-the-south-pole-and-north-pol}{This
post} was helpful (notation was taken from Lee Mosher's solution), viewing our
space as the quotient map $f: D_2 \to X$ performing exactly two
identifications. 2-disks can take different forms equivalent up to
homeomorphism and their boundaries can simply be circles. Seeing that $S^2$ is
simply $D_2$ attached along $e$ and $\bar{e}$, where $e$ is an arc connecting
the poles, was crucial.


\begin{exercise}[1.2.8]

Consider a surface $M_g$ of genus $g$ that is split into two closed surfaces
$M_h$ and $M_k$ by a circle $C$. $M'_h$ and $M'_k$ are constructed by deleting an open disk
from each. Show that neither $M'_h$ or $M'_k$ retract to
their boundary circle, but both (and all of $M_g$) retract to a lateral $C'$.

% https://math.solverer.com/library/allen_hatcher/algebraic_topology/exercise_1-2-9
% https://math.stackexchange.com/questions/479371/how-to-construct-a-genus-2-surface-from-8-gon
% https://analysis-situs.math.cnrs.fr/Classification-des-surfaces-triangulees-par-reduction-a-une-forme-normale.html

\end{exercise}

More familiarity with the two torus (really genus g surfaces in general) and
practice with abelianization is needed to tackle this problem.

\begin{figure}[ht!]
\centering
\includegraphics[width=90mm]{two-torus-hilbert.png}
\caption{The geometry of the two torus. Hilbert's \textit{Geometry and the Imagination} \label{overflow}}
\end{figure}

\begin{note}[Abelianization of $\pi_1(M_1)$]
	Consider $\pi_1(M_1) = < a, b ~|~ [a, b]$ as the familiar free group on two
	generators modulo the commutator (where $[a, b] = aba^{-1}b^{-1}$).

	The abelianization of $\pi_1(M_1)$ is then $\pi_1(M_1) \ [\pi_1(M_1), \pi_1(M_1)]$
\end{note}

\subsection{Covering Spaces}

Covering spaces let us (1) calculate fundamental groups of structures and (2)
think about algebraic properties using geometric intuition. (I do not really
understand how yet, so revisit this explanation after some exercise.)

\begin{definition}

	Given some space $X$, the space $\tilde{X}$, along with map $p: \tilde{X} \to
	X$, is a \textbf{covering space} if each $x \in X$ \textit{has some} open neighborhood
	$U$ of $x$, $p^{-1}(U)$ is a disjoint union of open sets where each one is
	mapped homeomorphically onto $U$ by $p$. These open sets are called
	\textbf{sheets}.
	
	$p^{-1}(U)$ is allowed to be empty, so $p$ need not be surjective.

\end{definition}

The helicoid (denoted $S$) is a good example that is easy to visualize. The helicoid can be
parameterized as: 

\[x = s \cos 2\pi t \]
\[y = s \sin 2\pi t \]
\[z = t \]

for $s \in (0, \infty), t \in \R$

\begin{figure}[ht!]
\centering
\includegraphics[width=90mm]{./helicoid.png}
\caption{}
\end{figure}

Then $p: S^1 \to \R \setminus \{0\}$ given by $(x, y, z) \mapsto (x, y)$ defines a
covering space of $\R - \{ 0 \}$.

I found this is best seen by looking at single line segments extending radially
from the origin. Consider $X = (0.5, 1.5) \times \{0\}$. $p^{-1}(X)$ is a
collection of disjoint segments. Our parameterization allows this segment of
the x axis to exist whenever we make one full rotation around the circle. There
are an infinite number of these segments and each maps to $X$ homeomorphically.

We can also look at the covering spaces of $S^1$. Let $p: S^1 \to S^1$ where
$p(z) = z \mapsto z^n$ with $z \in \Z$. Here $z \in S^1$ is the complex number
with $\abs{z} = 1$, so $S^1 = \{ z \in \C ~|~ \abs{z} = 1\}$.

Let $U = (0, \frac{\pi}{2})$ be some open segment of the circle, if $n=3$, what is
$p^{-1}(U)$?

Once place to see the utility of covering spaces is with oriented graphs.

\begin{figure}[ht!]
\centering
\includegraphics[width=90mm]{./2-oriented.png}
\caption{}
\end{figure}


Let $X$ (Fig. 3) be basepoint with two loops labeled $a$ and $b$. Now consider any
graph, $\tilde{X}$, where edge vertex has four edges. If we label, with $a$ or
$b$, and orient each edge we obtain a structure that looks like $X$ (respecting
orientation!) as we zoom into each vertex. We can call this an \textbf{2-oriented graph}.

If we construct a map $p: \tilde{X} \to X$, where the interior of each edge in 
$\tilde{X}$ maps homeomorphically to corresponding labeled edge in $X$
following the orientation of both edges, we satisfy the properties for a covering
space. It follows both that every oriented graph is a covering space of $X$ and
every covering space of $X$ can be 2-oriented (every vertex has 4 edges that
can be labeled so the local picture looks like $X$).

We are given a table of some of these $\tilde{X}$.

\begin{theorem}
	$\pi(X) \cong < a, b^2, bab^{-1} >$
\end{theorem}

\begin{proof}
Hatcher instructs us that we can use Van Kampen's.

% TODO - use the open sets from my notebook
\end{proof}

When we consider the fundamental group of the covering space as the "image
subgroup" of $\pi_1(X)$, $p_*(\pi_1(\tilde{X})) \leq \pi_1(X)$, we recover yet
another beautiful correspondence between algebra and geometry - there is a 
one to one correspondence between the subgroups $\pi_1(X)$ and covering spaces of $X$. 

A few more algebraic interpretations of covering spaces:

\begin{itemize}
	\item{If we change our basepoint in a covering space that is otherwise the
		same, $p_*(\pi_1(\tilde{X}, x_1))$ is a conjugacy subgroup of
		$p_*(\pi_1(\tilde{X}, x_0))$ in $\pi(X)$ where the conjugating element is the
	loop connecting the basepoints.}
\item{If a \textbf{symmetry} is an automorphism on a graph that preserves
		labels and orientations, then a graph can be "more" symmetric if it has
		more automorphisms and "the most" symmetric if every possible permutation
		of the graph vertexes have automorphisms with the desired properties. We
		will show that "the most symmetric" graphs are exactly those induced subgroups that
	are normal in $\pi_1(X)$.}
\end{itemize}

We now define three \textbf{lifting properties} of covering spaces and show
some applications. Given some covering space and a map into $X$, these
properties tell us when a lift of a homotopy exists, when a lift of a regular
map exists and when lifts are unique.

\begin{proposition}[Homotopy lifting property]
	Given a covering space $p: \tilde{X} \to X$, a homotopy $f_t: Y \to X$ and a
	lift of timepoint 0 of the homotopy $\tilde{f_0}: Y \to \tilde{X}$, there
	exists a \textit{unique} lift of the entire homotopy $\tilde{f_t}: Y \to X$.
\end{proposition}

When $Y$ is a point, this tells us that paths in $X$ induce unique paths in the
covering space.

When $Y$ is $I$, this tells us that homotopies in $X$ induce unique homotopies
in the covering space.

The key idea here is that these homotopies are unique. If we consider the lift
of a constant path, this implies that the lifted path is also constant 

\begin{note}
Check this. If $p\tilde{f} = f$ and $f$ is constant, if $\tilde{f}$ had anything
other than $\tilde{f}(0)$ in its image, what does that imply about the set of
allowed $\tilde{f}$s? If $\tilde{f}$ is itself constant, then $p\tilde{f} = f$
obvioulsy holds. 
\end{note}

We will use this to show $p_*$ is injective.

\begin{proposition}
	The induced homomorphism $p_*: \pi_1(\tilde{X}, \tilde{x_0}) \to \pi_1(X,
	x_0)$ is injective. The image subgroup $p_*(\pi_1(\tilde{X}, \tilde{x_0}))$
	consists of loops with baspoint $x_0$ that lift to loops in $\tilde{X}$ with
	basepoint $\tilde{x_0}$.
\end{proposition}

\begin{proof}
	To prove injectivity, consider two distinct homotopy classes, $[\tilde{f'_0}]$
	and $[\tilde{f''_0}]$, that have the same image under $p_*$, so
	$p_*([\tilde{f'_0}]) = p_*([\tilde{f'_0}])$. Indeed $p\tilde{f'_0} \simeq
	p\tilde{f''_0}$, given by some homotopy $f_t$, so our homotopy lifting
	property gives us a lifted homotopy $\tilde{f_t}$ where $\tilde{f'_0} \simeq
	\tilde{f''_0}$. Then $[\tilde{f'_0}] = [\tilde{f''_0}]$, so $p_*$ is
	injective.

	(This is a bit more explicit than Hatcher's proof using the kernel of $p_*$)

	A loop in $X$ with basepoint $x_0$ that lifts to a loop in $\tilde{X}$ with
	basepoint $\tilde{x_0}$ is by definition an element of $p_*(\pi_1(\tilde{X},
	\tilde{x_0}))$.

	The second statement is a bit bizarre, because should every element in the
	image subgroup, represented as $p_*\tilde{f}$, admit a straightforward lift
	$\tilde{f}$. I do not see where I have to invoke homotopy lifting, unless you
	consider some other representative of each homotopy class $g \simeq
	p\tilde{f}$, then $\tilde{g} \simeq \tilde{f}$, so the representative admits
	the desired lift.
\end{proof}

\begin{proposition}
	The number of sheets in a covering space $p$ is equal to the index of
	$p_*(\pi_1(\tilde{X}, \tilde{x_0}))$ in $\pi_1(X, x_0)$
\end{proposition}

\begin{proof}
	Let $g$ be a loop in $\pi_1(X, x_0)$. $\tilde{g}$ is its lift by the homotopy
	lifting property. Note that $\tilde{g}$ need not be a loop! Consider the lift
	of a single wrap around the circle in $p: \R \to S^1$.

	Each representative of our coset, where $H = p_*(\pi_1(\tilde{X}, \tilde{x}))$,
	$H[g] = h\cdot g$ has a lift $\tilde{h}\cdot\tilde{g}$. Notice that this lift
	is a path that starts at $\tilde{x_0}$ and ends at $\tilde{g}(1)$.

	We construct a bijection $\phi: H[g] \to p^{-1}(x_0)$ defined as $H[g] \to
	\tilde{g}(1)$. Because $\tilde{X}$ is path connected, for each point in
	$p^{-1}(x_0)$, we can draw a path to it from $\tilde{x_0}$. Notice that the
	projection of this path is always a loop at $x_0$ by definition of $p$. Then
	$\phi$ is surjective. If $\theta(H[g]) = \theta(H[g'])$, then $g(1) = g'(1)$
	and $g \cdot \bar{g'}$ lifts to a loop in $\tilde{X}$ with basepoint
	$\tilde{x_0}$. Then $[g \cdot \bar{g'}] \in H$ and $H[g] = H[g']$ ($\theta$ is
	injective).
\end{proof}

We have two more properties. One will show when a lift exists and another will
show when lifts are unique.

\begin{proposition}[Lifting criterion]
	Given a covering space $p: \tilde{X} \to X$ and some map $f: Y \to X$, there
	exists a lift of this map, denoted $\tilde{f}: Y \to \tilde{X}$, iff $f_*(\pi_1(Y, y_0)) \subseteq
	p_*(\pi_1(\tilde{X}, \tilde{x_0}))$.
\end{proposition}

\begin{proof}
	The necessary condition is straightforward. The existence of a lift implies
	that $f = p\tilde{f}$, then $f_* = p_*\tilde{f_*}$, so certainly
	$f_*(\pi_1(Y, y_0)) \subseteq p_*(\pi_1(\tilde{X}, \tilde{x_0}))$

	The sufficient condition is less straighforward. We proceed by constucting
	our lift $\tilde{f}$ explicitly. Let $\tilde{f} = y \mapsto \tilde{f\gamma}$,
	where $\gamma$ is any path from $y_0$ to $y$. We will show that this function
	is well-defined - that it does not matter what path we draw to $y$. 

	Consider another path $\gamma'$. We build a loop $\gamma \cdot \bar{\gamma'}$.
	Consider $f(\gamma \bar{\gamma'})$. By our assumption, this is some element
	of $p_*(\pi_1(\tilde{X}, \tilde{x_0}))$, call it $[ x ]$, so $f(\gamma
	\bar{\gamma'})$ homotopes to $x$. Because $x$ lifts to a loop with basepoint $\tilde{x_0}$, by the homotopy
	lifting property, $f(\gamma \bar{\gamma'})$ itself lifts to a loop with
	basepoint $\tilde{x_0}$. 

	We also know that this lifted path must be unique (by the same homotopy
	lifting property when considering a path as a homotopy of points, see the
	note above). Then $\tilde{f(\gamma \bar{\gamma'})} = \tilde{f(\gamma)} \cdot
	\tilde{f(\gamma')^{-1}}$. The equivalence comes from properties of the
	induced homomorphism and the lifts of the "half paths" each exist by the homotopy
	lifting property. The LHS is a loop so certainly the half paths must meet at
	their endpoints and $\tilde{f\gamma}(1) = \tilde{f\gamma'}(1)$ as desired.

	We must also check for continuity. Consider $y \in Y$, and let $U$ be a
	neighborhood about $f(y)$. Let $\tilde{U}$ be the sheet that covers
	$\tilde{f}(y)$ with a homeomorphism $p: \tilde{U} \to U$. 

	Let $N$ be a neighborhood about $y$ such that $f(N) \subseteq U$ by
	continuity. For any $y' \in N$, let $\gamma \cdot \eta$ be a path from $y_0$
	to $y'$, where $\gamma$ goes from $y_0$ to $y$ and $\eta$ goes from $y$ to
	$y'$. $\tilde{f}$ is defined by the endpoint of the lifted path $\gamma
	\eta$. But $\tilde{f \eta}$ is given by $p^{-1}f$ as it agrees with at least
	$\tilde{f y}$, the junction of $\gamma$ and $\eta$, so it cannot lie in some other sheet.

	Then restricted to $N$, $\tilde{f} = p^{-1}f$. This is true for
	arbitrary $N$ so $\tilde{f}$ is continuous.
\end{proof}

\begin{proposition}[Unique lifting property]
	Given a covering space $p: \tilde{X} \to X$ and some map $f: Y \to X$, if
	$\tilde{f_1}$ and $\tilde{f_2}$ agree on one point of $Y$ and $Y$ is
	connected, they agree on all of $Y$.
\end{proposition}

\begin{proof}
	For a point $y \in Y$ let $U$ be an evenly-covered neighborhood of $f(y)$
	(recall this means that $p^{-1}(U)$ is a set of open sets in $\tilde{X}$ each
	mapping homeomorphically to $U$). Let $\tilde{U_1}$ and $\tilde{U_2}$ be the
	open sets that $\tilde{f_1}(y)$ and $\tilde{f_2}(y)$ reside in respectively.

	If $\tilde{f_1}(y) \neq \tilde{f_2}(y)$, then $\tilde{U_1} \neq \tilde{U_2}$
	so these sheets are disjoint.

	Consider an open $N$ about $y$ that maps into $\tilde{U_1}$ and $\tilde{U_2}$
	by each lift. Such an $N$ exists because of continuity. If $\tilde{f_1}(y)
	= \tilde{f_2}(y)$, then $N$ must map into the same $\tilde{U}$. Because
	$p\tilde{f_1} = p\tilde{f_2}$ and $p$ is injective over over $\tilde{U}$ (it
	is homeomorphic with $U$), these lifts agree over $N$.
\end{proof}

\begin{definition}[semilocal simple connectedness]
	For each point $x \in X$, there exists $U$, where $\pi_1(U)
	\xhookrightarrow{}{\pi_1(X)}$ is trivial.
\end{definition}

Notice that this does not mean that $U$ itself is simply connected, rather that
loops within $U$ can homotope throughout the whole space to $X$. Local simple
connectedness is a much stronger condition.

\begin{definition}[local simple connectedness]
	For each point $x \in X$	the
	% TODO
\end{definition}

\begin{example}[Constructing the universal cover]
	We're going to build a topological space in quite an abstract way.

	Consider a semi-locally simply connected, path connected and semilocally path
	connected space $X$ (fucking mouth full - let us make sure we actually understand why
	each of these properties are needed throughout the proof and are not just
	listing them). We want to build a covering space, $p: \tilde{X} \to X$, that
	is simply connected.

	We first define the points of $\tilde{X}$ as $\{ [\gamma] ~|~ \gamma \text{ is a path in } X \text{ starting at } x_0\}$. To see why we are able to do this, we start with $p: \tilde{X} \to X$ that has already been constructed
	and is simply connected. For a basepoint $\tilde{x_0}$, each $x$ has exacly
	one homotopy class of paths because it is simply connected, call it $[ \gamma
	]$. These paths also form a homotopy class $[ p\gamma ]$ under the image of $p$, and by the homotopy lifting
	property, any member of $[ p\gamma ]$ must itself lift to or is homotopic to
	a path that lifts to $ [\gamma ]$. So the homotopy classes of paths in $X$
	and the points in $\tilde{X}$ are in one to one correspondence and we may use
	them interchangeably.

	We now must define a topology on $\tilde{X}$. Observe first that the
	collection $\mathscr{U} = \{ U ~|~ \pi_1(U) \xhookrightarrow{}{\pi_1(X)
	\text{ is trivial}}\}$ is a basis for $X$. Clearly it covers $X$, as $X$ is
	semilocal simply connected (esoteric property checked off), but observe also any $W \subseteq V \cap U \subset U$
	for $U, V \in \mathscr{U}$ also lies in $\mathscr{U}$ ($\pi_1(W) \xhookrightarrow{}{
	\pi_1(U)} \xhookrightarrow{}{\pi_1(X)}$ is trivial).

	We introduce some new notation. $U_{[\gamma]} = \{ [\gamma \cdot \eta] ~|~
	\eta \text{ is any path in } U \text{ where } \eta(0) = \gamma(1) \}$.
	Observe that $U_{[\gamma]} = U_{[\gamma']}$ iff $\gamma' \in [U_{[\gamma]}$, but
	\textbf{not} necessary if the endpoint just ends up in $U$ (see Hatcher for
	this quick proof).

	Then we build a basis $\mathscr{B} = \{ U_{[\gamma]} ~|~ U \in \mathscr{U}
	\}$.  Again $\mathscr{B}$ covers $\tilde{X}$ for the same reason, and for
	$\gamma'' \in U_{[\gamma]} \cap V_{[\gamma']}$ where $U_{[\gamma]},
	V_{[\gamma']} \in \mathscr{B}$, there exists $W \subseteq U \cap V$ where
	$W_{[\gamma'']} \subseteq U_{[\gamma'']} \cap V_{[\gamma'']}$. Our topology
	on $\tilde{X}$ is then generated by $\mathscr{B}$.

	Let $p: \tilde{X} \to X$ be defined as $\gamma \mapsto \gamma(1)$. $p$ is
	surjective as $X$ is path connected (another esoteric property checked off).
	$p$ is also injective as $p\gamma = p\gamma' \implies \gamma(1) = \gamma'(1)
	\implies \gamma \simeq \gamma'$. 

	For each sheet, $p: U_{[\gamma]} \to U$ is also a homeomorphism. 
	$p(V_{[\gamma]}) = V$ and $p^{-1}(U) \cap U_{[\gamma]} = V_{[\gamma'}$ (recall that the
	inverse images can be mapped into disjoint sets for each distinct homotopy
	class so the intersection is crucial).

	To see that $p^{-1}(U)$ is a collection of disjoint sheets. Notice that
	various $\gamma$ partition $U_{[\gamma]}$. Either $[\gamma] \neq [\gamma']$,
	so $U_{[\gamma]} \cap U_{[\gamma']} = \emptyset$. Else $[\gamma''] \in
	U_{[\gamma]} \cap U_{[\gamma']}$ then $U_{[\gamma]} = U_{[\gamma']} =
	U_{[\gamma'']}$. Notice that we get different sheets when the homotopy
	classes are different and we can start to get a feel for how eg. holes in $X$
	lead to different sheets in $\tilde{X}$.

	Our last task is to show that $\tilde{X}$ is simply connected. Because $p$ is
	injective it is sufficient to show $p_*(\pi_1(\tilde{X}, [\tilde{x}_0]))$ is
	trivial (check - why do we need $p$ to be injective?).
\end{example}

\begin{example}
	Let $X_{m,n}$ be the quotient space of a cylinder $S^1 \times I$ with the
identifications $(z, 0) \simeq (e^{\frac{2\pi i}{m}}z, 0)$ and $(z, 1) \simeq
(e^{\frac{2\pi i}{n}}z, 1)$. Note $z$ is a complex coordinate and the
transformation is a $\frac{1}{n}$ twist.

	$A$ and $B$ are subspaces of $X$ and the quotients of $S^1 \times [0, \frac{1}{2}]$ and $S^1
	\times [\frac{1}{2}, 1]$ respectively. One end of each half cylinder is just
	the circle ($A \cap B = S^1$), but they "twist" to the identification.

	See that if $m = n = 2$, then $A$ and $B$ are the mobius band.

	% TODO, see that both of them combined are the klein bottle
	% - Revisit the construction of non orientable surfaces by gluing disks onto
	% circles
	% - The boundary of a mobius strip is homeomorphic to circle
	% - In fact the boundary of each of 1.29 is a circle
	% - Walk through connection between projective plane and mobius strip
  %   Helpful! https://www.youtube.com/watch?v=B6vEdfk7SWU
	% ( https://www.groupoids.org.uk/outofline/motion.html#motion )

\end{example}

\begin{theorem}
	For each $H \leq \pi_1(X, x_0)$, there exists a covering space $p_*: X_H \to
	X$ where $p_*(\pi_1(X_H, \tilde{x_0})) = H$ for a suitably chosen basepoint
	$\tilde{x_0}$. 
\end{theorem}

So not only do we have a simply connected universal cover, but we have a cover
for every possible subgroup of our space. 

\begin{proof}

	Our strategy is to construct a quotient space out of our universal cover so
	that the points 

	First we define an equivalence class as $[\gamma] \sim [\gamma']$ if and only
	if $[\gamma \cdot \bar{\gamma'}] \in H$. We omit the proof of reflexivity,
	transitivity and symmetry as it is easy to see.

	We then take our universal cover $tilde{X}$ and identify points that belong
	in this equivalence class. Observe that this 

	For $[\gamma] = [c \cdot \gamma] \in H$ (where $[c]$ is the equivalence class
	of the identity), then $[c] \sim [\gamma]$.

\end{proof}

% https://www.groupoids.org.uk/outofline/motion.html#motion

\begin{exercise}[1.3.1]
	This is an exercise is basic definitions from point-set topology.

	For arbitrary $x \in A \subseteq X$, let $U$ be a neighborhood satisfying the
	covering space property when $x$ is considered a member of $X$. Namely, $U$
	is both open and $p^{-1}(U)$ is a collection of disjoint open subsets of
	$\tilde{X}$ that map homeomoprhically to $X$ by $p$.

	Then $U$ is open in the subspace $A$ by the definition of a subspace.
	If under the unrestricted map, $p^{-1}(U)$ is a collection of open and disjoint
	sets in $tilde{X}$ where each is homemorphic to $U$. It is easy to see that
	$p^{-1}(U \cap A)$ each is disjoint and has the desired homeomorphic
	correspondence. 

	%TODO - show openness
\end{exercise}

\begin{exercise}[1.3.3]

	Consider $p: \tilde{X} \to X$ where for each $x \in X$, $p^{-1}(x)$ is
	non-empty and finite.

	Show that $\tilde{X}$ is compact and Hausdorff iff $X$ is compact and
	Hausdorff.
\end{exercise}

\begin{proof}

	For the forward condition, consider $U, V$ about $x, y \in X$ respectively
	that are covered by sheets $\{U_{\alpha}\}$ and $\{V_{\alpha}\}$. Pick a pair of sheets
	$U_{\alpha}$ and $V_{\alpha}$. We can then identify separating neighborhoods
	$A$ and $B$ about $p^{-1}(x) \cap U_{\alpha}$ and $p^{-1}(y) \cap V_{\alpha}$
	in $U_{\alpha} \cup V_{\alpha}$ (in fact if $U_{\alpha} \cup V_{\alpha} =
	\emptyset$ then $A = U_{\alpha}$ and $B = V_{\alpha}$). This shows that $X$
	is Hausdroff.

	To see compactness, consider a cover $\mathcal{U}$ of $X$. For each $U \in
	\mathcal{U}$, we cannot say that it is covered by sheets, but we can say that
	each $x \in U$ has some $A$ that is covered by sheets. Then $\cup_x A_x$
	covers $U$. Repeating this procedure for each $U$, $\cup_U \cup_x A_x$ covers $X$ and $\cup_U \cup_x p^{-1}(A_x)$ is an open cover of $\tilde{X}$, as we chose each $A_x$ to be that which admits sheets under its preimage. $\tilde{X}$ is compact, so there exists a finite subcover $\mathcal{U'} \subset \cup_U \cup_x
	p^{-1}(A_x)$. $p(\mathcal{U'}) \cap \mathcal{U}$ is our finite subcover as
	desired.

	For the reverse condition, let $U$ and $V$ separate $p(u)$ and $p(v)$. Let
	$X$ and $Y$ be neighborhoods of $p(u)$ and $p(v)$ that are covered by sheets.
	Let $X_u$ and $Y_v$ be the sheets containing $u$ and $v$. Then $p^{-1}(U \cap
	X) \cap X_u$ and $p^{-1}(V \cap Y) \cap Y_v$ are disjoint open neighborhoods
	of $u$ and $v$ and $tilde{X}$ is Hausdorff.

\end{proof}

\begin{exercise}[1.3.4]
	Construct a simply connected covering space of the space $X \subset \R^3$ that is the union of
	the sphere and diameter. Do the same when $X$ is the union of the sphere and
	a circle that intersects it in two points.
\end{exercise}

\begin{proof}
	What is the simply connected covering space of a sphere?
\end{proof}

\section{Appendix}

\subsection{Geoemtry}

Practice with timeless objects that are the study of algebraic topology.

\begin{note}[Real projective plane]
	$\R P^2$
Awesome explanation. https://www.youtube.com/watch?v=B6vEdfk7SWU
\end{note}

\begin{note}[Genus-g surfaces]
Denoted $\Sigma_g$. $g$ tori strung together.
\end{note}

\subsection{Point Set Topology}

\begin{definition}[Basis]

	A basis for a topology $\mathscr{T}$ on $X$ is a collection $\mathscr{B}
	\subseteq \mathscr{T}$ such that every open set of the topology can be
	represented as a union of elements in $\mathscr{B}$.

	Equivalently, for each $U \in \mathscr{T}$ and $x \in U$, $\mathscr{B}$
	is the collection of $B$ where $x \in B \subseteq U$.

\end{definition}

It is straightforward to show these definitions are equivalent (check).

Bases then admit two properties:
\begin{itemize}
	\item{$\mathscr{B}$ covers $X$ ($X$ is certainly an open set that can be
		represented as a union of basis elements)}
\item{For each $x \in X$ and neighborhoods $B_1, B_2 \in \mathscr{B}$, there exists $B_3 \in
	\mathscr{B}$ where $x \in B_3 \in B_1 \cap B_2$ (As finite intersections of
open sets are open, we need basis elements to fit inside each intersection)}
\end{itemize}

\subsection{Presentations}

After more exposure to presentations to describe the group structure of some of
these geometric objects, it became clear that my understanding lacked rigor.

$G = <X~|~R>$ can be considered as the largest possible group with elements
that are words composed of the letters in $X$ \textbf{but} also subject to the
relations described in $R$. "Subject to the relations" means that we can swap
subwords for equivalent subwords or cancel subwords towards reducing different
words to the same underlying group element. 

(For $<a ~|~a^3>$, such reductions look like $aaaaaa = aaa = 1$, $(a^{-1}a^{-1})a = (a)a$)

I have found this hard to reason about the equivalence of words for
presentations (in abstract, not for particular groups) and it turns out that
this problem is very hard. Deciding that words are equivalent in $G$ just from
the generators and relations is unsolvable in general.

\begin{note}
	This problem is called the \textbf{word problem for groups}. In 1911, Max
	Dehn proposed that this problem was an important area of study for group
	theory, whereas prior most mathematicians used normal forms (irreducable
	representations) for group computations which made the word problem less
	relevant.
\end{note}

This does not mean that there do not exist groups where equivalences between
words are obvious. Consider again $<a ~|~a^3>$, any string / word, eg.
$a^4a^{-2}a^3\cdots$, can be reduced monotonically to one of $a, aa, aaa$. 

However, while this idea is intuitive, it is not satisfying, and a more formal
definition of the group describe by a presentation is desired:

\[ G = <X~|~R> = F(X) / <R>^{F(X)} \]

\begin{definition}
	The normal closure of $H$ in $G$ is the smallest normal subgroup containing
	$H$. It is computed by $< \{ xHx^{-1} | x \in G \} >$
\end{definition}

We will prove three properties of the presentation using this formal definition
\href{https://math.stackexchange.com/a/695061/1276086}{motivated by this post}.

\begin{proposition}
	G is generated by the images of $X$ in the quotient group.
\end{proposition}
\begin{proof}
Let $N = <R>^{F(X)}$. Clearly the generating set $\{ xN | x \in X \} \subseteq F(X) / N$. Then the
generated set $< \{ xN | x \in X \} > \subseteq F(X) / N$ as any word
$xNyN \cdots zN$ can be reduced to $(xy \cdots z)N$ and this word is in
$F(X) / N$.

To see the reverse, consider any $gN \in F(X) / N$ and observe $g$ is some word of
elements of $X$. Expand $g$ to the equivalent representation in letters of
$X$, $(xy \cdots z)N$ and see this element lies in the desired generated set.

We finish this proof by constructing an isomorphism by mapping each coset of $<
\{ xN | x \in X \}$ to the coset in that shares its representative $\subseteq
F(X) / N$.
\end{proof}

\begin{proposition}
	G is generated by the images of $X$ in the quotient group.
\end{proposition}

% There is a fundamental result expressing the universal property of 𝐺
% : if 𝐻
%  is any group, and 𝜙:𝑋→𝐻
%  is any map with the property that the images of the elements of 𝑅
%  under 𝜙
%  (you need to say exactly what that means) are all equal to the identity in 𝐻
% , then 𝜙
%  extends uniquely to a group homomorphism 𝐺→𝐻
% . This is not hard to prove from the definition of 𝐺
% .
% You should learn the basic Tietze transformations for manipulating group presentations. Again, the proof that these work can be proved easily from the definition.
% 
% 
% https://math.stackexchange.com/questions/3255411/universal-property-for-presentations

\begin{note}[Free Groups]
	See the \textbf{myasnikov.1.free.groups.pdf} for a category theory flavored exposition on free
	groups, their universal property and role in presentations.
\end{note}

\begin{note}[Universal Property of Free Groups]
Let $S$ be the generating set of a group $F(S)$. For any group $G$ and associated
map $f: S \to G$ (note $f$ is not a homomorphism and is just mapping symbols
into a group), $f$ extends to a unique (homomorphism) $f': F(S) \to G$ that commutes in the following diagram.

\[
\begin{tikzcd}
S \arrow[r, "f"] \arrow[d, "i"'] & G \\
F(S) \arrow[ru, "f'"']           &  
\end{tikzcd}
\]

We proceed with a brief proof and then an example.

If we consider $F(S)$ as reduced words in letters of $S$, let $f' =
f(s_1)\cdots f(s_n)$ and claim this is our extended homomorphism.

Uniqueness can be seen becauase $f'$ must agree with $f$ for each $s \in S$.

\end{note}

Lets construct two isomomorphisms between presentations and groups to understand
what a quotient of this enormous free group and almost just as enormous normal
subgroup actually mean in the context of groups we already understand.

\begin{itemize}
	\item{$\Z \times \Z \cong <a,b~|~[a,b]>$}
	\item{$\S_3 \cong <a,b~|~a^2=b^3=aba^{-1}b^2>$}
\end{itemize}

\begin{proposition}
	$D_3 \cong <x,y~|~x^3=y^2=xyx^{-2}y>$
\end{proposition}

\begin{proof}

	Lets begin by invoking the universal property:
	\[
	\begin{tikzcd}
	{\{x, y\}} \arrow[rd, "f"'] \arrow[r, "i"] & {F(x, y)} \arrow[d, "\phi"] \\
																						 & D_3                        
	\end{tikzcd}
	\]

	Where $F(x, y)$ is our free group on two generator, $i$ is the inclusion map,
	$f$ is a map from our generators to letters in the dihedral group. The
	diagram commutes ($f = \phi i$).

	By definition, our presentation is $F(x, y) / << x^2, y^3, xyx^{-2}y >>$
	(where the $<<->>$ denotes the normal closure). I claim that our universal
	$\theta$ gives us the desired isomorphism when restricted to the quotient
	group.

	Let us expand our commutative diagram by factoring $\phi$ through the
	quotient group:

	\[
	\begin{tikzcd}
	{\{x, y\}} \arrow[rd, "f"'] \arrow[r, "i"] & {F(x, y)} \arrow[d, "\pi"]   \\
																						 & D_3 / N \arrow[d, "\phi'"] \\
																						 & D_3                         
	\end{tikzcd}
	\]

	Where $pi = w \to w << x^2, y^3, xyx^{-2}y >>$ projects words onto
	their cosets and $\phi' = wN \to \phi(w)$. Notice $\phi'$ is well defined as
	the image of every element of the normal closure is the identity ("$\phi$
	respects the relations").

	Then $\phi'$ is surjective as $\phi$ is surjective (every element of $D_3$ is
	word of letters in the image of $f$). To see $\phi'$ is injective, consider
	distinct cosets $aN \neq bN$. Then $\phi'(aN) = \phi'(a)\phi'(N) = \phi'(a) =
	\phi(a)$. Similarly, $\phi'(bN) = \phi(b)$. $\phi(a) \neq \phi(b)$

\end{proof}


\begin{note}
	\textbf{Tietze transforms} define operations on presentations that do not
	change the generated group. Adding and removing both generators and relations
	are allowed if the generator or relation can be derived from other
	information in the presentation.
	$< x, y ~|~ x^2 = y = 1>$
\end{note}

\subsection{Derived Subgroups}

Commutators and derived subgroups appear frequently to describe the structure
of fundamental groups of cell complexes. We will review some definitions and
proofs from elementary algebra.

\begin{definition}
	The \textbf{commutator} of two elements $a, b \in G$ is $a^{-1}b^{-1}ab$ and
	is denoted $[a, b]$.
\end{definition}

\begin{definition}
	$a^x$ is the conjugate of $a$ by $x$ ($xax^{-1}$).
\end{definition}

Clearly the commutator is the identity element if $a$ and $b$ commute in $G$
(hence its name). We derive some pedagogical identities that will be useful.

\begin{definition}
	$[a, b]^x = [a^x,b^x]$
\end{definition}
\begin{proof}
	$xa^{-1}b^{-1}abx^{-1} = xa^{-1}x^{-1}xb^{-1}x^{-1}xax^{-1}xbx^{-1} = [a^x,b^x]$
\end{proof}

\begin{note}
	The set of commutators need not be closed over the group operation.

	Consider the free group on four generators. Then $[a, b][c, d] =
	a^{-1}b^{-1}abc^{-1}d^{-1}cd$. This is not a commutator in this group (there
	are no two elements $x, y$ such that $[x, y] = [a, b][c, d]$.
\end{note}

Even though products of commutators are not themselves commutators in general,
the subgroup \textit{generated} by all commutators in a group is certainly closed over
the group operation. This leads naturally to the derived subgroup.

\begin{definition}
	The group generated by all of the commutators in $G$ is the \textbf{derived
	subgroup} of $G$, denoted $[G, G]$ or $G'$.
\end{definition}

\begin{theorem}
	$[G, G] \trianglelefteq G$
\end{theorem}

\begin{proof}
	$([a, b] \cdots [e, f])^x = ([a^x, b^x] \cdots [e^x, f^x])$
\end{proof}

\begin{theorem}
	If the quotient group $G / N$ is abelian, then $[G, G] \subseteq N$. In other
	words, the derived subgroup the smallest subgroup that abelianizes $G$.
\end{theorem}

\begin{proof}
	Let $N$ be some normal subgroup of $G$, then $N$ abelianizes $G$ iff for each
	$x, y \in G$, $xyN = yxN$. Then $N = y^{-1}x^{-1}yxN$ (the group operation is
	well-defined as $N$ is normal) so $y^{-1}x^{-1}yx \in N$. 

	Certainly the set of commutators must be included in $N$. So any minimal $N$
	is the minimal group generated by these commutators. But this is exactly $[G,
	G]$.
\end{proof}

\begin{definition}
	$G / [G, G]$ is the abelianization of $G$ and sometimes denoted $G^{ab}$.
\end{definition}


\subsection{Continuity}

I found myself confused about the equivalence between the $\delta \epsilon$
definition of continuity from calculus and the topological definition using
open sets. The direction of the implications is what is not obvious.

Recall the conceptual meat of continuity is that small perturbations in the
range can always be . 

More specifically the preimage of some small interval in the range
always exists in the domain if we

\begin{definition}
	A function is continuous at $x$ if for all $\epsilon$ there exists an
	$\delta$ where
	\[\abs{x - x_0} < \delta \implies \abs{f(x) - f(x_0)} < \epsilon\]
\end{definition}

The structure of this implication is crucial. We can construct arbitrarily
small intervals of the range and always recover some $\delta$ about $x_0$ that
entirely maps within this interval. Simple piecewise discontinuous funcions
from analysis illustrate how clearly discontinuous maps (they have big holes)
map onto this delta epsilon definition. We will walk through one for the sake of
pedagogy.

\begin{example}
Consider
\[ \begin{cases} 
      x^2 - 3 & x\leq 0 \\
			\sin x & x > 0
   \end{cases}
\]

Consider continuity at $x = 0$. For $\epsilon = 2$, there exists no value of
$\delta$ where $\abs{x - 0} \implies \abs{f(x) - f(0)} < 2$.
\end{example}

% https://math.stackexchange.com/questions/2823758/why-is-the-topological-definition-of-continuous-the-way-it-is

Continuity is defined differently in topology:

\begin{definition}
	Let $f: X \to Y$ be a continuous map. Then for each open $U\in Y$,
	$f^{-1}(U)$ is open in $X$.
\end{definition}

\subsection{Complex Numbers}

% TODO

$z = a + bi$ can be represented as

$e^{\theta i} = \sin\theta + i \cos\theta$

\end{document}
