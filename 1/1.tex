\documentclass[10pt]{article}
\usepackage{kennyworkman}

\newcommand\restr[2]{{% we make the whole thing an ordinary symbol
  \left.\kern-\nulldelimiterspace % automatically resize the bar with \right
  #1 % the function
  \vphantom{\big|} % pretend it's a little taller at normal size
  \right|_{#2} % this is the delimiter
}}

\title{Hatcher: Algebraic Topology}
\author{Kenny Workman}
\date{\today}

\begin{document}

\maketitle

\section{The Fundamental Group}

\subsection{Intuition}

\textbf{The Borromean rings} are three linked circles where any two circles without a third are unlinked. It is a nice object that elucidates the difference between abelian and nonabelian fundamental groups.

Label the three rings $A$, $B$, and $C$. Consider $C$ as an element of $\R^3 \setminus A \cup B$. $C$ can be represented as $aba^{-1}b^{-1}$, where $a$ and $a^{-1}$ are forward and reverse oriented paths around $A$ (and $B$). This element is not trivial, so the free group generated by $a$ and $b$ is not abelian.

Modify the ring so that $A$ and $B$ are linked. The same $C = aba^{-1}b^{-1}$ is now trivial. It is easiest to see this visually. Refer to the picture in Hatcher and slide the first loop $a$ all the way around $A$ until it is hanging over the circle. This loop is clearly contractible.

From this, we conclude there are two ways to tell if two circles are linked:

\begin{itemize}
	\item{Show that one of the circles is an element of the fundamental group of the complement of the other circle. (Simple example of loop wrapping around a circle.)}
	\item{Show the fundamental group of the complement of both of the circles is the abelian. (As in the Borromean ring example.)}
\end{itemize}

\subsection{Basic Constructions}

\begin{definition}[Path]
A path in the space $X$ is a continuous map $f: I \to X$ where $I$ is the unit
interval $[0, 1]$.
\end{definition}

\begin{definition}[Homotopy of paths]

A continuous family of maps represented $f: I \times I \to X$ 

A path in the space $X$ is a continuous map $f: I \to X$ where $I$ is the unit
interval $[0, 1]$.
\end{definition}

\begin{example}[Linear homotopy]
Consider two paths $f, g$ that share endpoints ($f(0) = g(0) = x_0$ and $f(1) =
g(1) = x_1$). Any such paths are always homotopic by the homotopy 
\[
h_t(s) = (1-t)f(s) + (t)g(s).
\]

% TODO: show continuity with vector argument

\end{example}


\begin{proposition}[homotopy of paths with fixed endpoints is an equivalence relation]
\end{proposition}

\begin{definition}[The product of paths]

	For paths $f, g$ where $f(1) = g(0)$, we can define the product $h = f \cdot g$ as

\[ h(s) = \begin{cases} 
      f(2s) & 0 \leq s \leq 0.5 \\
      g(2s-1) & 0.5 \leq s \leq 1
   \end{cases}
\]

\end{definition}


\begin{proposition}[The fundamental group is in fact a group]
	$\pi_1(X, x_0)$ with respect to the operator $[f]\cdot[g] = [f\cdot g]$ is a group.
\end{proposition}

\begin{definition}[Reparamterization of a path]
	We can precompose any path $f$ with $\phi: I \to I$ such that $f\phi \simeq f$. $f\phi$ is then the reparameterization of $f$.
\end{definition}

\begin{proof}
	To see our operator is well-defined, $[f\cdot g]$ depends only on $[f]$ and $[g]$. In other words, a homotopy exists between paths in $[f\cdot g]$ if and only if it is the composite of homotopies between paths in $[f]$ and $[g]$.

	First, we verify associativity of the operator. Consider $(f \cdot g) \cdot h$. This product composes the composition of $f$ and $g$ with $h$. Visually, $h$ is the first half and $g$ and $f$ occupy the last quarters of the product path. We can construct a piecewise function to change the proportions of these components - "shrink" $h$, keep $g$ the same and "expand" $f$. $[(f \cdot g) \cdot h]\phi \simeq f \cdot (g \cdot h)$ gives us the desired equivalence.

	Second, we check the identity property of the constant map $c$. $f$ can be reparameterized by the piecewise function that speeds it up for the first half of $I$ and holds it constant at $f(1)$ for the second half where $f\phi \simeq f \cdot c$.

	Third, we verify the existence of an inverse for each $[f]$. Define $[\bar{f}]$ where $\bar{f}(s) = f(1-s)$. Then $f\cdot\bar{f}$ is homotopic to the constant path. To see this construct $h_t = f_tg_t$ where $f_t = f$ on $[0, t]$ and $f_t = f(t)$ on $[1-t, 1]$ and $g_t$ is the inverse of this function. As $t$ approaches $1$, $h_t$ has a larger constant region in its image starting from the middle and growing to the endpoints until $h_1(s) = f(0) = g(1)$ is our constant map at timepoint 1.

\end{proof}

\begin{proposition}[The fundamental group is independent of the choice of basepoint up to isomorphism]
	If $f$ is a loop with basepoint $x_1$ and $h$ is a path from $x_0$ to $x_1$. The map $\beta_h: \pi_1(X, x_0) \to \pi_1(X, x_1)$ defined as $\beta_h([f]) = [h \cdot f \cdot \hat{h}]$ is an isomorphism.
\end{proposition}
\begin{proof}
	$\beta_h$ is well-defined. To see this, if $[f] = [f']$ implies that $f$ and $f'$ are homotopic, so certainly $h \cdot f \cdot \bar{h}$ and $h \cdot f' \cdot \bar{h}$ are homotopic and $\beta_h([f]) = \beta([f'])$.
	$\beta_h$ is a homomorphism. $\beta_h([f \cdot g]) = [h \cdot f \cdot g \cdot \bar{h}] = [h \cdot f \cdot \cdot h \cdot \bar{h} \cdot g \cdot \bar{h}] = [h \cdot f \cdot \cdot h] \cdot [\bar{h} \cdot g \cdot \bar{h}] = \beta_h([f]) \cdot = \beta_h([g])$. 
	$\beta_h$ is an isomorphism.

\end{proof}

\begin{definition}
	A space is \textbf{simply connected} if it is path connected and has a trivial fundamental group.
\end{definition}


\begin{theorem}[$\Z \cong \pi(S1)$]

The map $\phi$ sending the integer $n$ to the homotopy class of the loop $\omega_n(x) = (cos2\pi nx, sin2\pi nx)$ on $S^1$ with basepoint $(1, 0)$ is an isomorphism.

\end{theorem}

\begin{proof}

Observe that each loop $\omega_n$ can be expressed as the composition $p\tilde{\omega}_n$, where $p(s) = (cos2\pi s, sin2\pi s)$ and $\tilde{\omega}_n(s) = ns$ is the path in $\R$ from $0$ to $n$. $\tilde{\omega}_n$ is called the lift of $\omega_n$.

We define the image $n \in Z$ under $\phi$ as the homotopy class represented by $pf_n$ where $f_n$ is any path in $\R$ from $0$ to $n$. Since any such $f_n$ shares endpoints with $\tilde{\omega}_n$, the paths are homotopic, and $pf_n$ and $p\tilde{\omega}_n$ are also homotopic. Then $\phi(n) = [pf_n] = [w_n]$.

We quickly verify $\phi$ is a homomorphism by considering $\phi(m + n)$ as $p\tilde{\omega}_{m+n}$

We now define two facts that show $\phi$ is a bijection:
\begin{itemize}
	\item[(a)]Given a path $f$ in $S^1$ starting at $x_0$, there exists a unique lift of the path $f$, $\tilde{f}$, in $\R$ starting at $\tilde{x_0}$ for any choice of $\tilde{x_0} \in p^{-1}{(x_0)}$.
	\item[(b)]Given a homotopy $f_t$ in $S^1$ of paths starting at $x_0$, there exists a unique lifted homotopy $\tilde{f_t}$ of paths in $\R$ starting at $\tilde{x_0}$ for any choice of $\tilde{x_0} \in p^{-1}{(x_0)}$.
\end{itemize}

$(a)$ gives injectivity of $\phi$. Consider any loop $f$ with basepoint $(1, 0)$ reperesenting a homotopy class in $\pi_1(S^1)$. Then there exists $\tilde{f}$ starting at $0 \in p^{-1}((1, 0))$ that ends at $n \in \Z \subset p^{-1}((1, 0))$. So $\phi(n) = [f]$

$(b)$ gives surjectivity of $\phi$. Consider $\phi(m) = \phi(n)$. Then the representative loops $\omega_m$ and $\omega_n$ are homotopic. Define this homotopy as $f_t$. $(b)$ gives us $\tilde{f}_t$ where $\tilde{f}_t(0) = 0$ for all $t$ (Our choice of $\tilde{x_0}$). $\tilde{f}_0(1) = m$ and $\tilde{f}_1(1) = n$, but the endpoint of the homotopy must be invariant with respect to time, so $m = n$.

We observe that both of these facts are specific examples of a more general fact $(c)$

\begin{itemize}
	\item[(c)] For an arbitrary space $Y$, given $F: Y \times I \to S^1$ and a lift $\tilde{F}: Y \times \{0\} \to S^1$ of $\restr{F}{Y \times \{0\}}$, there is a unique lift $\tilde{F}: Y \times I \to S^1$ that restricts to the given $\tilde{F}$ on $Y \times \{0\}$.
\end{itemize}

Observe that $(a)$ follows trivially when we consider $Y$ to be a point. 
To see that $(c)$ implies $(b)$, observe the homotopy of paths in $S^1$, $f_t$ given in $(b)$ connects $f_0$ and $f_1$. $(a)$ gives us a unique $\tilde{f_0}: I \times \{0\} \to S^1$, a restriction of the lifted homotopy, and $(c)$ gives us a unique lifted homotopy $\tilde{f_t}$ that restricts to $\tilde{f_0}$. Note that $\tilde{f_0}$ defines the endpoints of homotopic paths in $\R$ and is unique so $(b)$ follows.

% Revisit after JC
% We prove $(c)$ by explicitly constructing $\tilde{F}: N \times I \to \R$ for some neighborhood $N$ of any $y_0 \in Y$ and showing that $\tilde{F}(y_0, 0)$ uniquely determines $\restr{\tilde{F}}{\{y_0\}\times I}$. Then $\tilde{F}$ must be unique, and because $y_0 \in N \subset Y$ was chosen arbitrarily, we obtain a unique lift over all of $Y \times I$.
% 
% We will use that fact that an open cover $\{ U_{\alpha}\}$ of $S^1$ exists in our argument. In particular, that $p^{-1}(U_{\alpha})$ is a set of disjoint open sets, each of which homeomorphic to $U_{\alpha}$ for each element of this covering.
% 
% First, we construct $\tilde{F}: N \times I \to \R$ where $N$ is some neighborhood of $y_0 \in Y$. We can choose open sets $(y_0, t) \in N_t \times [a_t, b_t]$ for each $t$ where $F(N_t \times [a_t, b_t]) \subset U_{\alpha}$ for some $U_{\alpha}$. Then we can choose small enough $N$ where $F(N \times [a_t, b_t]) \subset U_{\alpha}$ for some $U_{\alpha}$, namely $N = \cap_t N_t$. 
% 
% We partition $I$ into intervals $[t_i, t_{i+1}]$ and use induction to define. Recall we are given $\restr{\tilde{F}}{Y \times \{0\}}$ and use this define $\tilde{F}$ on $[0, t_1]$. Then $\tilde{F}$ on $[t_i, t_{i+1}]$ is

\end{proof}

We can immediately use $\pi_1(S^1)$ to prove important theorems. The big idea is that each full rotation around the circle is a unique element of the group. This suggests, among other things, that a loop once around cannot be continuously deformed to a loop twice around.

Our arguments proceed by contradiction, by assuming that our desired result does not hold and showing that this assumption implies some homotopy between loops on the circle that is not allowed.

\begin{theorem}[Fundamental Theorem of Algebra]
Every non constant polynomial with coefficients in $\C$ has at least one root in $\C$.
\end{theorem}

\begin{proof}
	Assume our polynomial $p(z) = z^n + a_1z^{n-1} ... a_n$ has no roots.
	Define the following $f_r: I \to \C$ for each real number:
	\[
		f_r(s) = \frac{p(re^{2\pi is}) \backslash p(r)}{\abs{p(re^{2\pi is})} \backslash \abs{p(r)}}
	\]
	We claim this is a loop in the unit circle $S^1$. To see this, compute some values for fixed $r$ and note for any value of $r$, $f_r(0) = 1$, $f_r(1) = 1$.
	Note that $f_0$ is a constant map with value $1$ and is the trivial loop. Then by varying $r$ we obtain a homotopy between any $f_r$ and $0$ with (the embedded) basepoint $(1, 0)$.
	\indent We now show that $p$ must be constant. Choose a large $r$ such that $r \geq \sum_n{\abs{a_n}} \geq 1$. Observe then $r^n = rr^{n-1} \geq (\sum_n{\abs{a_n}})r^{n-1}$.
	This expression motivates a new homotopy for each $f_r$, let $f_{(r, t)}$ be our previous definition with $p_t(z, t) = z^n + t(z^{n-1}a_1 + ... a_n)$ substituted for each $p$. See that $f_t$ has no zeros on the circle of complex values with radius $r$ satisfying our expression (this is why we constructed it at all).
	Note that $f_{(r, 0)} = e^{2\pi stn}$ (check). Then $f_{(r, t)}$ is a homotopy between $f_r$ and homotopy class $[w_n] \in \pi_1(S^1)$.  But $f_r$ is homotopic to $0$. Then $w_n$ is also homotopic to $0$ and $n$ must be 0. Our $p(z) = a_n$ must be constant.
\end{proof}

\begin{theorem}[Brouwer fixed point theorem]
	Any continuous map $h: D^2 \to D^2$ must have a fixed point $h(x) = x$.
\end{theorem}

This theorem was initially proved by a gentleman named L.E.J. Brouwer circa 1910 and seems to be foundational to the rest of algebraic topology and other fields like differential topology.

We now introduce a theorem which proves that there must exist two places on the surface of the earth with both the same temperature and pressure.

\begin{theorem}[Borsuk-Ulam theorem]
	A continuous map $f: S^2 \to \R^2$ must have at least one pair of antipodal points, $f(x) = f(-x)$.
\end{theorem}

Lets build intuition for the two dimensional case by proving the theorem in one dimensions. For a map $f: S^1 \to \R$, construct continuous $h = f(x) - f(-x)$. Pick a point $x$, and evaluate $h$ at points $x$ and $-x$ halfway across the circle from each other. Notice $h(x) = -h(x)$, then there must exist some $y$ where $h(y) = 0 \in [h(x), -h(x)]$ by intermediate value theorem. We proceed with proof in two dimensions:

\begin{proof}
	Define a path $\eta(s) = (\cos 2\pi s, \sin 2\pi s, 0)$ around the equator of $S^2$ and a map $g: S^2 \to S^1$ defined as $g(x) = \frac{f(x) - f(-x)}{\abs{f(x) - f(-x)}}$. Let $h = g\eta$. Check that $\eta$ is nullhomotopic (one can shrink the equator to the origin) and therefore $h$ is also nullhomotopic. 

	\indent We now examine the lift $\tilde{h}$ of $h$ to see that $h$ cannot be nullhomotopic. Observe $g(x) = -g(-x)$ so $h(s) = -h(s + \frac 1 2)$ for $s \in [0, \frac 1 2]$. Then $\tilde{h}(s + \frac 1 2) = \tilde{h}(s) + \frac q 2$ for some odd integer $q$. (to visualize this, observe $h(s)$ and $-h(s + \frac 1 2)$ are on opposite sides of $S^1$, so the distance between them in the lift is a path in $\R$ that is an arbitrary number of full loops and one half loop). Then $h$ lies in the homotopy class represented by the generator of $\pi_1(S^1)$ times a nonzero integer q, so it cannot be nullhomootpic.
\end{proof}

This theorem says a few things. It tells us the surface of a sphere can never be one-to-one with an embedding in $R^2$, so it cannot be homeomorphic with a subspace of $R^2$. Think of trying to produce such a map with the surface of a sphere, $S^2$, and the plane - one would have to introduce a hole somewhere.


\begin{theorem}[Fundamental group of a product space]
	$\pi_1(X \times Y) \cong \pi_1(X) \times \pi_1(Y)$ if $X$ and $Y$ are path connected.
\end{theorem}

\begin{example}[Torus]
	The fundamental group of the torus can be thought of as a pair of integers. Formally, $\pi_1(S^1 \times S^1) \cong \Z \times \Z$. 
\end{example}


\begin{definition}[Induced homomorphism]
	The map $\varphi: X \to Y$ where $\varphi(x_0) = y_0$ induces a homomorphism $\varphi_*: \pi_1(X, x_0) \to \pi_1(Y, y_0)$ defined as $\varphi_*([f]) = [\varphi f]$
\end{definition}

We can briefly check some properties of this homomorphism. Let $\varphi, \psi$ be maps:
\begin{itemize}
	\item{$(\varphi\psi)_* = \varphi_*\psi_*$}
	\item{$\mathds{1}_* = \mathds{1}$}
\end{itemize}
These properties of the induced homomorphism make the fundamental group a functor.

The induced homomorphism allows us to describe relationships between topological spaces as relationships between fundamental groups.

\begin{proposition}[]
	For a given retraction $r: X \to A$, the induced homomorphism for the associated inclusion $i_*$ is injective. If $r$ is a deformation retraction, $i_*$ is also an isomorphism.
\end{proposition}

\begin{proof}
	Because $ri = \mathds{1}_A$, $r_*i_* = \mathds{1}_{A*}$ so $i_*$ is injective. To see $i_*: \pi(A) \to \pi(X)$ is also surjective, see that any loop in $X$ homotopes to a loop in $A$ by $r_tf$ if $r_t$ is a deformation retract.
\end{proof}

	An important takeaway from the above is that spaces that are continuous deformations of each other, like those that are deformation retracts of each other, have in some loose sense, the same number of homotopy classes of loops. Their fundamental groups are isomorphic.

	However, deformation retracts are rather strict examples of homotopies as they restrict to the identity map over their retracting space for all $I$ ($\restr{r_t}{A\timesI} = \mathds{1}$). In particular, any deformation retract fixes a basepoint between $X$ and $A$.

	Consider a homotopy $\varphi_t: X \times I \to Y$ that is not a deformation retract but fixes the basepoint in $X$ ($\varphi_t(x_0) = y_0$ for all $t$). Then $\varphi_{0*} = \varphi_{1*}$ as $[\varphi_0f] = [\varphi_1f]$. Any pair of induced homomorphisms under a basepoint preserving homotopy are equivalent.

	Now consider homotopy equivalences that are also basepoint preserving. Let $\varphi: X \to Y$ and $\psi: Y \to X$ be such equivalences. Then $\psi\varphi \simeq \mathds{1}$ but we just saw then $(\psi\varphi)_* = \mathds{1}_*$ so $\varphi$ is injective. $\varphi$ can be shown to be surjective and thus an isomorphism by a symmetric argument.

	We can actually see this is true even if our homotopy equivalence does not fix our basepoint.

\begin{lemma}[]
	Let $\varphi_t: X \to Y$ be a homotopy and $\restr{h}{x_0 \times I}$ be a path between $\varphi_0(x_0)$ and $\varphi_1(x_0)$. Then $\varphi_{0*} = \beta_h \varphi_{1*}$
\end{lemma}

\begin{note}
	To see this, for a given loop $f$ in $X$, we can build a new homotopy $\bar{h_t} \cdot \varphi_tf \cdot h_t$. Then $\beta_h(\varphi_0([f]))$ is homotopic to $\varphi_1([f])$.
\end{note}

\begin{theorem}[The choice of basepoint between homotopy equivalent spaces is not important]
Consider the homotopy equivalence $\varphi: X \to Y$, then $\pi_1(X, x_0) \cong \pi_1(Y, \varphi(x_0))$.
\end{theorem}

\begin{proof}
	Consider the homotopy inverse $\psi: Y \to X$. Then $\psi\varphi \simeq \mathds{1}$. We just saw that $(\psi\varphi)_* = \beta_h(\mathds{1})_*$ for some path $h$ from $\psi\varphi(x_0)$ to $x_0$. Then $\psi_*\varphi_* = \beta_h$ is an isomorphism and $\varphi_*$ is injective.
	Using the same argument, $\varphi\psi \simeq \mathds{1}$ and $\varphi$ is also surjective.
\end{proof}

\begin{exercise}[]

\end{exercise}
\begin{proof}
\end{proof}

\begin{exercise}[1.1.3]
	$\pi_1(X)$ is abelian iff the basepoint-preserving homomorphism $\beta_h$ depends only on the choice of endpoints of $h$
\end{exercise}

\begin{proof}

	Consider homotopy classes $[f], [h]$ in $X$. By definition, $\beta_h([f]) = [hfh^{-1}]$. Consider a new constant path $c = s \mapsto h(0)$ (the basepoint of $X$). Then if $\beta_c = \beta_h$, $[f] = \beta_c([f]) = \beta_h([f]) = [hfh^{-1}] = [h][f][h^{-1}]$ for any choice of $f, h$. $\pi_1(X)$ is abelian.

	If $h(0) \neq h(1)$, and $h(0) = g(0)$ and $h(1) = g(1)$, then $\beta_h = \beta_g$ are equivalent because paths with the same endpoints are homotopic ($[f] = [hfh^{-1}] = [gfg^{-1}]$). If $h(0) = h(1)$, this is only true if $\pi_1(X)$ is abelian.

\end{proof}

\end{document}
